%!TEX root = ../crimson_throne_book_main.tex
% 2014-01-19
Having followed one of the rebellious dockworkers to a secret meeting in a warehouse, the young heroes have discovered that someone is actually instigating the riots. A brief but tough fight left most of the rabble-rousers unconscious, although Puk's vicious blades made one casualty that cannot be saved. The main instigator, a dashing bard by the name of Feldon, failed to get away when Quint's {\itshape sleep} spell took him down. The companions tie him up and decide to interrogate him before they call in the guard. At first Feldon seems fervent in his belief that the queen is incompetent. Her lack to take control over the last few hectic days only proves this point. Quint wants to know who Feldon is working for, but although the man is quite chatty, he is also extremely vague. He lets on that he might be more willing to cooperate if his captors provide him with an outlook on a future. To say it plainly, if he's handed over to the guard, his fate will be sealed. He wants a way out. It takes two broken knees (apparently this is turning into Sjo's trademark interrogation technique) before the squirming prisoner can convince the companions that he will only speak if they guarantee not to hand him over to the guard. If the heroes give him their word - especially Sjo, who clearly demonstrates his faith in the law and in 'a word given' - he is willing to stay in their custody until the problems in the city quiet down. He's even prepared to give his own word not to try and escape during this time, if he'll be allowed to go free afterwards. In exchange he can provide valuable information or so he claims. After some deliberation the heroes agree.\\

Then Feldon confesses: he is working for someone who has no political agenda at all. His employer only has his own economic interests at heart. By destabilizing Korvosan society his boss hopes to plummet real estate prices, so he can buy up huge portions of the city and acquire a position of note in Korvosa. Feldon's help in this scheme is equally inspired by greed as the bard is handsomely paid for his services. With some luck he might even be able to copy his employer's plans on a smaller scale and invest in some Korvosan houses or businesses. The man Feldon works for does not hail from Korvosa, but from Cheliax: he's the Chelaxian ambassador Amprei!\\

While most of the dockworkers are still unconscious from the fight, one of them was defeated with a {\itshape sleep} spell, just like Feldon. This man was tied up and awoke shortly after the battle, before the companions started interrogating their prisoner. He has heard every word the honey-voiced bard has confessed to, and now he fumes with anger. Quint and Sjo use some charges of the  {\itshape cure} wand to bring the man's fallen comrades back to consciousness and when they all realize they have been played by a money-grubbing Chelaxian dandy, they are equally enraged. Now they understand that they have been risking their lives so someone else could get rich off their blood. Instead of wanting to march against the queen and convince others to join them, they now want to stop as many Korvosans as possible from rioting for a misguided cause. This insight more or less turned them from misled enemies of the state into defenders of the peace, so the companions decide to let the men go, as they are now bent on preventing tomorrow's demonstration. Feldon is taken to the old fishery and locked up in Gaedran Lamm's old room. He explains that the Vudran beauty Selena is someone like him, who was inspiring another cell of instigators elsewhere in the city. Those people will have the same agenda as the one Feldon was trying to give to the dockworkers. Tomorrow at noon, they will gather their supporters and head for the Gold Market for a violent demonstration. Balian remains behind to guard the prisoner while his friends make for Citadel Volshyenek to inform Field Marshal Kroft of the impending uproar in the city's commercial center.\\

Cressida Kroft is asleep behind her desk when Sjo, Quint and Puk arrive. She is shocked to learn that ambassador Amprei has been feeding the riots for his personal gain, but she immediately states that it is impossible for anyone but the queen to act openly against the Chelaxian emissary. She will have to consult with some people to see what can be done about the man. Until that time the companions should not take action against Amprei. Still, priority has to go to tomorrow's demonstration. Kroft will inform her officers and work out a plan to prevent the manifestation. It looks like another night without sleep for her ...\\

\section{23 Desnus 4708}

When the companions wake up the next morning, they feel well rested and even more equipped to tackle today's challenges (level up to level 3). When to head to the market they find it completely locked off by the Korvosan Guard. Anyone trying to get in is turned down and sent away. Quint and his friends join some men who have come from the Shingles and return home after the guards refused to let them pass. A fellow called Evald, who normally hangs around in {\itshape The Blue Drake} , told them that today would mark a turnaround in Korvosan history as the people would demand to be heard in a massive manifestation in the Gold Market. Apparently someone snitched on these plans to the guard, as the soldiers were not supposed to be in the loop. More disappointed marchers have gathered in the Blue Drake, but there is no sign of Evald. The innkeeper knows where he lives, though, in a flat two blocks away. Quint leads his friends there, but finds only a young frustrated woman, who is annoyed that her husband is out rioting again. She has no idea where Evald is and actually gets worried when she hears that he is 'missing'.\\

Back in the Blue Drake Quint inquires about Selena. The innkeeper confirms that Evald has met with a Vudran lady a couple of times, although he doesn't know her name. She was quite a looker, though. A few hours later Evald shows up in his favorite haunt. Quint gets to talking with him, informing him that people in the docks are telling strange tales: the riots are apparently caused by someone who wants to throw the city into chaos to undermine the real estate market and buy up huge portions of the city. Tales about the queen's incompetence were fabricated to fuel an evil scheme of a power-hungry villain, who was using Korvosan citizens as propellant. He also adds that a woman named Selena is one of the evildoer's agents. Evald is shaken; as the news sinks in, he gets furious at having been deceived, a sentiment that is quickly shared by other patrons in the tavern.\\

Seeing that their rumor spreading is so successful, Quint and his companions spend the rest of the day in the Shingles, feeding the gossip channels, dissuading more people from supporting the anti-queen movement. They make sure to mention the Vudran snake in the grass, Selena, as an accomplice of the unknown manipulator. After all, she's not the only one who can sway minds with words.\\

