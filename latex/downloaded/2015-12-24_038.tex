%!TEX root = ../crimson_throne_book_main.tex
% 2015-12-24
\section{11 Arodus 4708}

When the party rises the next morning, Krojun and his Burn riders have already left camp to go auroch hunting. They have taken a number of Skull clan warriors with them. The heroes spend the day in the company of their young friend Lerrim, and Fartshek, another youngster who turns out to be Thousand Bones grandson -- or son-son, as he calls it himself. While the tribe's shamans are preparing a hut for tonight's spirit consulting ritual, Fartshek tells the visitors from Korvosa more about the legend that is Krojun Eats-What-He-Kills.\\

The imposing tribesman is the son of Chief Ready-Klar of the one of the numerous Sun clan tribes. He is one of the Sklar-Quah's most fearsome warriors, specializing in the klar and earthbreaker technique. As a young men he saw his tribe suffer several brutal attacks at the hands of orc champion Kyrust Chief Killer, a Rotten Tongue marauder from Urglin. Desperate to help his people, Krojun sought the aid of a reclusive Shoanti sorcerer who lived alone deep in the Mindspin Mountains. The hermit sent Krojun on several punishing tests, promising him that if he succeeded he would earn a great reward that would help him defeat Kyrust. The tests were harrowing indeed, designed in part to train Krojun in the ways of the Thunder and Fang fighting style, and it took Krojun many months to complete them. In the end, he stood before the sorcerer in triumph. When Krojun demanded his reward, however, the sorcerer responded only that he had no reward to give and vanished. Krojun's rage was great, and when he returned to his people empty-handed, he found that his entire tribe had been enslaved by Kyrust. Krojun tracked the slave caravan for days, finally catching up to it a few miles from Urglin's gates, and in a fantastic display of rage and power, single-handedly defeated the orcs and their leader Kyrust. It was only as Krojun claimed the orc's belt of giant strength as both a trophy and a symbol of the Sklar-Quah's power over the orcs that he realized the truth--that strange old sorcerer had indeed given him a gift: the gift of rage. Without the skills and strength Krojun honed in completing the tasks the sorcerer had set him to, he would surely have fallen in such a combat as he had just won. Today, Krojun is at the forefront of the Sun Clan's efforts to strike back at the orcs and tshamek who have hammered away at the Shoanti for centuries.\\

Sjo also wonders about the strange golden-hued wingless dragon creature he has seen roaming around camp. Right now the wonderful being is lying on a big rock, enjoying the warmth of the sun. Fartshek explains that this is Wicked-Claws, a dragon-like beast known as a dragonne. It lost its wings in a fight with a ferocious bulette some years ago, before chief One-Life saved it from certain death. No longer able to soar the skies above, Wicked-Claws adopted the Skull clan as her family and now protects the tribe from danger.\\

By the time the sun sets Krojun's hunting party still hasn't returned yet. All preparations for the ritual are ready now. Thousand Bones, Ashdancer and jothka One-Life invite the visitors to strip down and join them in a small hut that radiates waves of heat. Smoldering hot rocks have been heaped up in the middle of the tent. Once everyone has found a place around the stones, Ashdancer takes a large wooden spoon and pours water on the hot rocks. Scented steam fills the hut and immediately everyone breaks out in a sweat.\\

Thousand Bones hums in a low tone and then speaks: {\itshape``Tonight we consult our ancestors. Great spirits from the past, show us whether we can truly trust these {\itshape tshamek} and show us what we have to do to achieve peace for both their people and ours.}'' Ashdancer joins Thousand Bones in his humming and then changes the tune to a mesmerizing tribal chant. Ever so often she spills another spoon of water over the searing stones, filling the hut with boiling smoke. One by one the companions succumb to the heat and slip into unconsciousness. Sjo suddenly finds himself in a lush, green meadow and sees that his arms have turned into the mighty forelegs of proud red brown stallion. A massive, noble tree draws his attention. As he trots up to the stately plant, he sees other animals approaching. A nimble puma darts through the high grass and without hesitation Sjo knows it is Puk. Then he sees the snout of a fox peaking up which is undeniably Quint. A few moments later Balian arrives as well, in the shape of a massive bear.\\

Walking under the tree Sjo sees a black panther lying lazily on one of the branches: Ashdancer. Two birds swoop in as well, an owl and an eagle: Thousand Bones and One-Life. Looking around Sjo realizes that there are other animals about, not in real physical form, but in an outline of light blue energy. These must be the spirits of the forefathers. It also dawns on the healer that this world does not inhibit his sight and he can see well into the horizon.\\

The owl turns its head around and regards the heroes in their animal forms with its milky-white eyes. {\itshape``The ancestors find in you no treacherous intent towards my people, since you all appeared as animals of the natural world}'', he hoots. {\itshape``Now behold what they have to tell you!}'' He takes to the air again and swoops closely over the companions. The next moment he is gone, as are Ashdancer and One-Life. The majestic tree suddenly pulls away to the horizon and mountains come into view, from where a light rumble swells into a loud thundering. The earth itself starts to tremble when a herd of wild horses appears over the closest hilltop. Some of the mounts have small round plaques of gold dangling from their necks, others bears medallions of silver, while others bear a dreamcatcher-like emblem made of tiny bones. An imposing stallion clad in multiple golden circles leads the herd. Having reached the last hill, the animal pulls to a stop, rears and whinnies loudly. When its hooves hit the ground, sparks shoot up. The rest of the herd start trampling impatiently. Quint wonders what the horses are looking at. He shifts his gaze across the plain and sees a large gray mass making its way through the long grass. One wolfish howl fills the air, quickly joined by the overwhelming howl of massive pack of wolves. The horses leap forward and storm into the plain, racing at the charging pack of canines. A few heartbeats later the two armies tear into each other. Wolves growl and snap at everything equine, while the horses use their hard hooves to kick around, showing no mercy for their enemies. The battles turns into a veritable bloodbath with no one gaining the upper hand.\\

Then a deafening bang burst from the dark clouds at the horizon. Wolf and horse cease their animosity for a second, before a shockwave reaches them and throws them all to the ground. The clouds pull together and swoop closer, slowly taking the shape of a terrifying dragon made of shadow, her wings are a living curtain of darkness that falls over everything within sight and swallows the complete landscape. The battlefield disappears in blackness and then the dark sweeps over the companions, drawing them in an endless plummet into the great nothing. As he falls, the big bear that is Balian, suddenly sees a small sparrow batting its wings furiously, trying to catch up and grab a hold of his fur. It bites and scratches in Balian's shoulder, struggling to hold on, but then it loses its grip and it is forced to let go. As it tumbles away in the dark, a humongous hyena mouth appears on top of it and swallows it whole. Then Balian's fall suddenly ends as he jumps to his feet, having returned in his own body inside the steam hut. His friends awake as well.\\

Thousand Bones asks the companions what they witnessed. When they tell him about the horses and wolves, he believes they represent the Shoanti and Korvosans going to war. The shadow dragon could be Ileosa or whatever evil that is controlling her. The vision proves to him that waging war with Korvosa will end badly for both sides. When Balian informs the others about the sparrow clinging to his fur, he notices that there are tiny scratches in his shoulder, forming Shoanti runes. They spell out: {\itshape``Saamesh Urgir help}''.\\

The ranger has heard of Urgir, a great orc city controlling the lands known as the Hold of Belkzen. Quint recalls stories of a legendary orc warlord Belkzen who managed to unite all the orc tribes under his banner and lead his people to victory over the dwarf army stationed at the sky citadel Koldukar. After crushing the dwarf army in the Battle of Nine Stones Belkzen seized control of the mighty citadel and renamed it Urgir, meaning {\itshape``first home}'' in Orcish. He expanded his holdings and constructed a small empire for his people which survived the centuries and lasts until today. No successor of Belkzen ever succeeded in truly uniting all orc tribes again, making the Hold of Belkzen a hodgepodge of dozens of orc tribes, all vying for power. The most powerful tribe controls the city of Urgir and by extension the rest of the orcish lands, but this control is tenuous at best.\\

Thousand Bones has more to add. He knows what the word 'Saamesh' refers to, it is the name of a young Shoanti warrior from the Sun clan who resides in the Hold of Belkzen as a spy. Much like the Korvosans to the south, the orcs of the north are the Shoanti's archenemies and it is Saamesh' task to evaluate the threat. Saamesh is not just any Shoanti either, he is the youngest son of Valkur Burns-in-his-Veins, the chief of the biggest Sklar-Quah tribe and therefore the first among all Sun clan jothka's. Thanks to Saamesh' intelligence, the Shoanti have learned more about the current leader of Urgir, Grask Uldeth of the Empty Hand. This mighty orc warchief has been at the head of the Hold for over a decade now, making him somewhat of an exception in a long line of orc rulers. He has proven himself a much more clever leader than his predecessors, because unlike them he is surprisingly accommodating to foreigners in Urgir. He allows humans, dwarves, elves, gnomes and halflings into the city to trade, because he has come to realize that wealth from trading with non-orcs is more profitable than simple raiding. This progressive thinking is frowned upon by many of his kin, but none of them can deny the wealth the {\itshape pinkskins} have brought to Urgir. Thousand Bones believes the spirits have granted the heroes this information to allow them to go to Urgir and save Saamesh, who obviously needs help. It would give the party the chance to gain the Sun clan's favor, for the Shoanti crede says that you can be accepted either {\itshape``by blood or deed}''. Anyone who is not of pure Shoanti blood can still become a 'nalharest', a brother, by proving himself worthy through his actions. Thousand Boens has already accepted the companions as his 'nalharest', but his only extends to his own tribe. If the party wants to attain the same level of recognition with the Sun clan, they'll have to find a new way to achieve that. This Saamesh business might just be the perfect opportunity.\\

The old shaman also tells his guests that in order for 'pinkskins' to get into Urgir, they need the patronage of one of the many tribes' chiefs. Only when carrying a token or badge from a tribal orc commander can minority races exist in Urgir. These badges signify that the bearer has the protection and permission of that tribe to reside within Urgir's limits and consequently provides some safety in the orc city, but it does not avoid mistreatment and discrimination. To facilitate the party's travel to the distant lands of Belkzen, Thousand Bones offers to prepare a teleportation ritual, which will be ready by tomorrow evening. For now he bids them good night.\\

When the heroes return to the camp, they see that Krojun has returned from a successful trip. A big roast is being prepared while warriors are boasting their prowess in today's hunt. Krojun bursts into a hearty laugh when he notices the visitors. {\itshape``Well, if it isn't the lowlanders. How are we today? No stiff neck? I must admit that I missed you at the hunt. I'm curious to see how you would fare against the mighty aurochs. Seeing the auburn manes of these great beast waves like flames today really reminded me of a great Sun clan tradition, the Burn Run, our infamous rite of passage. Brave youngsters have to prove their mettle by outracing a wildfire, bolting for the safety of the great Yondabakari River before they are overcome by smoke or consumed by flames.\\

If only there were a river and grasses here, then I would like to see you prove yourself in your own little Burn Run. But no such luck, although we could swap out the fire for the auroch mane. I have never tried to outrun an auroch stampede, but I rejoice at the thought. What do you say? Are you up for the challenge of a special auroch run?}''\\

