%!TEX root = ../crimson_throne_book_main.tex
% 2013-09-21
\section{26 Gozran 4708}

Since Isha is one of the few the companions haven't interrogated yet with regard to Lick's sudden illness, they decide to look her up. She is still asleep in the infirmary of the temple of Sarenrae. The conversation with Lick's girlfriend confirms the friends' suspicions that Isha is innocent. When she learns that Lick is in the cathedral of Pharasma, the girl panics, afraid that her boyfriend has passed away. Pharasma is the goddess of death after all. Learning that Lick is still alive, she asks Balian and Sjo to transport him to his flat in Old Korvosa, hoping that his own bed might soothe whatever is ailing him.\\

Meanwhile Quint decides to examine Lick's older work, to see if his writings have always been so brilliant. If only his latest play was so exceptional, he might have made a deal with a devil or something to receive inspiration to write one 'superb' play. This infernal muse might have left him afterwards, explaining why there is an 'emptiness' inside him. This theory is even confirmed when Sjo recalls Lick's harrow reading. The mad bard allowed only one card to be read about him: the tyrant, a mighty blue dragon that clutches the world in its claws. According to Zellara it stood for a mighty force exerting control over the bard which could cause him great harm. Sjo remembers that Lick was distraught after hearing this message.\\

Quint pays another visit to Pilts Swastel's theatre and spends the rest of the days reading through the five plays Lick has written over the last year and a half. Although there is certainly some progression in his work, it has always been good from a literary point of view. His subjects however are consistently less high-classed, ranging from true horror in his first production to the more macabre and sexually inspired play the companions witnessed last Sunday. Still, nothing points to a sudden leap in quality.\\

Sjo goes to Zellara's parlor in Lancet Street, where he summons the fortune-teller's spirit from the Harrow deck. Zellara cannot provide more insight into the tyrant card she drew for Lick. The card pointed in a certain direction, as she unfolded when she drew it. It did not give her the power to look into Lick's head, however, so the card's precise meaning remains hidden to Zellara as well. She does support the theory that some kind of muse might have left Lick, leaving an 'emptiness' behind.\\

\section{27 Gozran 4708}

Quint takes his mates to the university of Korvosa to examine what Lick was researching there. A friendly history professor welcomes the young men and guides them through the library. His name is Cyril Fordyce, husband to Eliasia Leroung, the current headmistress of the university. Professor Fordyce is very forthcoming, disclosing that Lick spent close to three months between the books, reading up on any source about the Shoanti. He even inspired the professor to write a series about the years of struggle with the Shoanti tribes for the Korvosa Herald (the paper reserves its backside for the 'History of Korvosa'). The next issue will feature another of those 'Korvosa versus the Shoanti' episodes.\\

Only one of the books actually mentions the Shoanti 'Donodarr'. He must have been quite a hero among his people, slaying over a dozen of settlers in the early days of the conflict. Still the book describes him as a formidable menace, as it was written from the point of view of Donodarr's enemies, the Chelish settlers. Nothing suggests that Donodarr was a demon, although the ferocity with which he fought was fired by a monstrous rage.\\

Professor Fordyce also clarifies that there are no books written in Shoanti. The barbarians used their runic script only in stone, wood or tattoos, since they used no paper or parchment. Most of their writings can be found in the old burial mounds that used to dot the landscape or on their skins. One of the professor's wife's ancestors, the famous explorer Montlarion Jeggare, recorded the runic Shoanti language in a book, however. Sjo uses it to decipher the runes on the wooden board that he was found with as a baby, and arrives at 'Shaoban', so it puzzles him as well why his name should actually be pronounced Shee-buhn.\\

At the end of the day the companions go back home. Three days of research and questioning people have provided a hint of insight into Lick's illness, but nothing they discovered points to a possible source. Maybe they should just accept that some things are beyond their power to understand or unearth.\\

\section{28 Gozran 4708}

On the morrow of the 28th of Gozran, Isha turns up at the fishery's doorstep. Her eyes are red from crying. When she sees Quint and the others, tears start running down her face once more. Lick, she sobs, is dead.\\

While she was sleeping last night, he must have woken up from his delirious slumber and snuck out of the apartment. He broke into the Exemplary Execrables theater and hanged himself above the main stage. The last couple of days Lick has mostly been unconscious. The rare moments he was awake, he was ailing with fever. Isha remained by his side night and day, until she passed out from exhaustion last night. When she woke up this morning, she found Lick's bed abandoned. Not much later, an employee from Exemplary Execrables came by to bring her the terrible news. Pilts Swastel, the owner of the city's creepiest theater, had made a macabre discovery: Lick dangling from the rigging at the end of a rope!\\

