%!TEX root = ../crimson_throne_book_main.tex
% 2014-01-11
\section{22 Desnus 4708}

The first order of the day is getting a new {\itshape wand of cure light wounds} , as the current one is almost depleted. Sjo knows that the best place to find such an item is the temple of Sarenrae. On the steps of the Sun Goddess's church two parties are locked in a violent discussion with a priest. A couple of soldiers are carrying an unconscious colleague on a stretcher. Facing them is a tear-faced mother and her wounded 12-year-old son. The boy is clinging his left arm closely to his body, trying to prevent his broken bones from hurting too much. "In the name of the law, I demand you see to my man!" the sergeant of the Korvosan Guard commands.\\

"But my son, he's so young ... he's hurting so much, the poor thing can hardly breathe. Please heal him, we're devote followers of Sarenrae", the woman sobs.\\

"Young, you say? He was old enough to cause a riot and throw stones at my men! If you heal him instead of my soldier, I'll arrest the lot of you for disturbing the queen's peace!" the sergeant bellows as his hand slides towards the hilt of his blade.\\

Seeing how hard pressed the priest is, Sjo steps in. "There is no need for a fight. There is enough healing to go around, don't fret." Nathan, the priest, recognizes the former acolyte of Sarenrae and his face lights up. Sjo gauges the situation and quickly establishes that the boy's wounds need to be treated sooner than the soldier's. If the broken arm is not set right, it won't heal properly and leave the kid handicapped for the rest of his life. The soldier might have lost consciousness, but his wounds are not severe. He urges the priest to heal the boy, while he tends to the soldier with a {\itshape cure light wounds} of his own. Balian is the only one who sees through the weeping mother's deceit. Although she claims to be a true follower of Sarenrae, she sports a medallion of Asmodeus around her neck. Still, Sarenrae's priests are the only ones who hand out healing for free, so she has come to the home of the Sun Goddess instead. Nathan still helps her by giving her son his last cure spell of the day and hopes that this charity will make her see Sarenrae's value.\\

When the guards, the mother and son have left, Nathan invites the companions in. He is really pleased to see Sjo. Although he still doesn't quite understand where the big man gets his divine powers from, he figures it has something to do with his Shoanti heritage, giving him a shaman's skills rather than a priest's. Even so, he claims that this mystifying source of Sjo's powers should not prevent him from praying to Sarenrae. He also explains that he and his fellow priests have been very busy helping the wounded. As Sarenrae is the only church in town that heals people for free, the clergy has been hard pressed by requests for medical aid. As the priests only hand out their daily allotment of spells free of charge, they usually go through their curing arsenal pretty swiftly. Most people who are turned down because there is no more healing left, return early the next morning, which leaves most priests depleted of healing magic by the time the sun has properly risen. That also explains why he had only one cure spell left for the two wounded parties. Nathan takes his guests to see Iris, who handles the commercial transactions. She happily sells Sjo a fully charged {\itshape wand of cure light wounds} . Quint and Sjo also wonder what the queen is up to. Since she hasn't really shown her presence over the last couple of days and she hasn't made any decisions to solve the chaotic situation, they fear there might be trouble. Quint would like to procure an audience with the monarch to offer his help, but doesn't know how to proceed to get in to the castle to see her. So he decides to write a letter to the captain of her bodyguard, Sabina Merrin. In it he explains what he and his friends have done and learned over the last few days and offers his services to the city and the crown, urging her to contact him. He hands the letter to one of the Gray Maidens who are guarding the palace. He also notices a strong presence of Hellknights in the streets surrounding the castle, which explains why this area is so quiet.\\

Sjo suggests paying a visit to the Gray District, since he would like to find out what happened to the seneschal. Neolandus Kalepopolis disappeared during the first hours of unrest, but nobody seems to have a clue what happened to the man. Sjo questions the priests of Pharasma to find out whether the seneschal's corpse might have been brought in and buried. Keppira d'Bear assures him that she has not seen Kalepopolis' body among the victims of the riots. She knows the seneschal in person and unless his cadaver was so badly damaged that it was unrecognizable, she is sure he was not buried with the nameless deceased.\\

Sjo also establishes that the Shoanti are still here. Gaekhen explains that their diplomatic mission has not changed, but the visitors from the Cinderlands are waiting for a more opportune time to approach the queen. Since the feud between his people and Korvosa is buried in three centuries of blood and hatred, it would not be wise to be too hasty. The conflict is three hundred years old, a few more weeks won't matter. Sjo also questions Thousand Bones about his take on the riots and the presence of the castle on top of the mastaba. The old shaman does not understand the situation. Although the Shoanti know both leadership by family ties and leadership by strength, all parties involved are always open and upfront about their intentions. The people of Korvosa act incoherently and no strong leaders have presented themselves and taken control. As far as the castle is concerned, he regrets this insult to his people, as the mastaba is a holy place for Shoanti, but he realizes that he has to accept this fact if he wants to negotiate with the city's leaders.\\

