%!TEX root = ../crimson_throne_book_main.tex
% 2013-09-07
After lunch Sjo leads his friends to the cathedral of Pharasma, picking up Lerrim on the way. A friendly young priestess leads them to two tents behind the church, where a dozen of proud warriors greet them. Their leader is an old shaman. The tall Shoanti is easily over 60 winters old and leans heavily on his walking stick, the polished femur of some giant beast crowned with a cougar's skull. His shirt is adorned with countless jangling animal bones. A bearskin cloak hangs about his bony shoulders and gray warpaint in the shape of a skull decorates his face. His eyes are even milkier than Sjo's, giving the impression that he is blind. A red-feathered crow sits on his shoulder. The shaman speaks with a deep voice and despite his white eyes, he looks at his guests as if he sees not only them, but their souls as well.\\

"Kel-grish, my name is Thousand-Bones, shaman of the Skoan-Quah, the clan of the Skull. This is my daughter-husband Fergal the Fang and his sons, my daughter-sons Arowan, Erogar and Gaekhen. How can we be of service to you?"\\

Sjo and Quint explain their plight: they freed young Lerrim recently from a child abuser, but the boy is still traumatized and does not speak, in fact he hasn't spoken in three years.\\

Gaekhen approaches the boy, looks him in the eye and says: "Storval dharanok ekbit roark Shoanti", which Quint recognizes as a typical greeting among Shoanti. It means 'thunder rolls over the Storval plateau'. To everyone's surprise Lerrim responds: "Ador roark sklar", 'and so rolls our fire', which identifies him as a member of the Sklar-quah, the clan of the Sun.\\

Thousand-Bones takes the boy apart to talk to him, while Sjo produces the only item he inherited from his Shoanti parents: the wooden board with his name. Gaekhen reads it: Shaoban. He says it is also a Sklar-Quah name, so Sjo is probably from the same clan as Lerrim, or at least one of his parents was, since Sjo seems to have more than Shoanti blood running through his veins. Gaekhen explains that the correct pronunciation of the name Shaoban is not 'Show-buhn', but 'Shee-buhn'. Sjo is confuses, Gaedran Lamm always claimed that his name had to be pronounced with the /ow/-sound and not with an /ee/. Quint is quick to point out that they had better not change his name from Sjo to She, since that would sound ridiculous.\\

Some time later Thousand-Bones returns with Lerrim. The boy has told him his story: three years ago he'd gone out hunting with his father. They were attacked by the Cinderlander, the scourge of the Sklar-Quah, a man of Chelish blood who dedicated his life to hunting down Shoanti. He shot Lerrim's father, decapitated him and planted his head on a pike, sticking two crossbow bolts through his eyes to prevent his soul from traveling to the happy hunting-grounds. Then he took young Lerrim to the city of Kaer Maga, where he sold him at 'Flesh Block', the slave market, to a Korvosan merchant. The merchant quickly learned that Shoanti do not make good slaves and traded the boy to Gaedran Lamm upon arriving in Korvosa.\\

Lerrim did not speak the language of the Korvosans at first, but he rapidly understood that keeping quiet got him into a lot less trouble than cursing and swearing in Shoanti. So the boy taught himself to shut up and continued doing so, even when he started to grasp the strange new language.\\

Thousand-Bones thanks the young heroes for their kindness toward one of his people and considers it an honor to look after the boy and return him to his clan. He also explains why he is here: his clan, the Skoan-Quah, are looking for a peaceful way for the Shoanti and the Korvosans to co-exist. Since his forefathers were driven out of the lowland more than two centuries ago, the people of Korvosa and his Shoanti brethren have stayed arch-enemies. Many Shoanti, especially the Sklar-Quah, who used to live on these shores, are still looking for revenge, while the Korvosan continue to hate the 'barbarians' like no-one else. Unfortunately peace is still far off, but the old shaman does not give up hope for a better future.\\

The Skoan-Quah are far more than Shoanti's diplomats though. They are the caretakers of the dead, first and foremost, among all Shoanti. This also explains Thousand-Bones' good relations with the church of Pharasma in Korvosa, since they have been entrusted with similar tasks. The clan of the Skull also track the oral history of their people, since they travel so freely among other clans and stress the importance of keeping the memory of the honorable dead alive.\\

Learning that the Skoan-Quah remember the stories of their forefathers, Quint asks the shaman whether he has ever heard of Donodarr. A friend performed some kind of ritual on stage to awaken the spirit of a Shoanti demon by that name. Since then he has been in a feverish delirium as if possessed. Thousand-Bones has indeed heard of Donodarr, although he was hardly a demon. Donodarr was one of the earliest warriors to lead the Sklar-Quah against the Chelish invaders on these shores. He died defending his homeland and was most likely buried somewhere on the mainland. It is possible that his spirit has not found peace and now haunts Quint's friend. The shaman promises to examine Lick if his friends manage to transport him to the Gray District.\\

