%!TEX root = ../crimson_throne_book_main.tex
% 2014-04-26
\section{4 Sarenith 4708}

The second day of rehearsals requires only three actors on stage. The rest of the cast is present as well and watches as Aisha, Dario and Marco practice their parts. Aisha impresses again and Dario is excellent as well, despite his rotten real-life character. Illusionist Rimando Lumenos spoils the director's morning by summoning the image of a luscious Shoanti babe instead of a threatening assassin. The gnome's sense of humor does not blend well with the director's nervous nature and need for control. Marco Ebhart becomes a true and convincing wildman while playing the Shoanti assassin, but during lunch break he airs his disdain of the Shoanti race, making him clash with Sjo.\\

Dierderik Lodann, the ginger-haired merchant who plays Alika's father is a vessel of gossip. He points out that Ali\"ena Fochs, the lovely woman who portrays the heavenly angel at the end of the play, has a son who is anything but an angel. He thought himself quite the rebel and got arrested during the riots. He is still in jail. Arianna Evenland, captain Keyra Palin in the play, is even more intimately familiar with prison life, as she spent fifteen years in the slammer for murder. Lodann does not know how she avoided the death penalty, but implies it might have something to do with her former profession, a lady of the streets. Of course, not every killing qualifies as murder, so that might have something to do with it as well.\\

Lumenos continues his pranks during lunch time. He hands Puk and Sjo the end of a piece of rope and asks him to hold it so he can measure something. He disappears with the rope around the corner. After they've been standing around with the rope in their hands for a couple of minutes, the halfling learns that Varric Bedan has been holding the other end of the rope for minutes already. The rogue decides to step in on the joke and has Varric and Sjo pull the rope tight, so they almost destroy the stage. Still, bringing Varric and Sjo together has a positive effect, as the veteran from the Guard has performed on stage before and teaches Sjo and Balian how to act out a fight on scene.\\

When normal rehearsals are over, Quint and Sjo pick up their singing lessons, convincing diva Auralia Lazanne to train them as well. In the evening Quint drops by the university to see professor Fordyce. He is allowed to read Alika Epakena's real journals, which are far less glorious than the events shown in {\itshape The Passion of Saint Alika} . They mostly describe the hard life of early settlers who had to learn to survive in a hostile environment. The professor also suggests that Quint becomes a formal student at the university, but unless he is allowed to skip class and study by himself, the bard's life will not mix well with a student's. \section{5 Sarenith 4708}

The third act is the longest of the opera and has lots of people on stage. Quint is up again and he performs excellently in the opening number, in which he mocks Dario Darnas's character for believing in Alika's visions. Palastus nags endlessly about where everybody has to stand on stage and constantly shouts at Leiny to make notes.\\

When Alika receives her vision of a water source, Dario Darnas is supposed to pick up an iron pole and hit it in the ground, but the rod slips from his grasp, spoiling his performance. It turns out Lumenos greased the thing, eliciting some smiles from the cast when Darnas makes a fool of himself. Aisha seems especially amused. The gnomish prankster continues his jests when he has to conjure up the image of a lightning bolt that hits the iron pole. He has the flash of electricity hit Darnas instead, garnering another burst of anger from Palastus, but the secret appreciation from so many others in the cast.\\

After lunch the second part of act three is on. Varric Bedar still struggles with his voice and gets quite a verbal beating from the director. Since Quint only features in the first half of the third act, he is in the audience again. Ali\"ena Fochs seeks him and Sjo out and repeats the story they learned from Lodann yesterday. He son Alex, who is only fifteen, was indeed arrested for rioting and is in jail awaiting trial. She claims he is a good boy who runs around the wrong crowd. Like so many others he was misled into protesting, destroying some property and insulting the Queen and the Guard. Still, he did nothing unforgivable: he didn't wound or kill anybody. Since Quint and Sjo know Field Marshal Cressida Kroft personally and both seem to be very nice guys, she wonders if they could talk to the commander of the Guard on her son's behalf. Quint can't make any promises, but gives his word to do his best.\\

