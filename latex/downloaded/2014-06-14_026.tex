%!TEX root = ../crimson_throne_book_main.tex
% 2014-06-14
After he has dragged away Trinia from the edge of the roof, Balian somewhat loosens his grip on the girl. Meanwhile Quint tries to calm her down, for there is fear in her eyes. She picked up the rumor that both the guard and a lynch mob were looking for her because she supposedly poisoned the king, an accusation she blatantly denies. Quint confirms that he and his friends have seen no proof whatsoever that she is guilty and suggests going back to Nisha's flat to talk, as he is worried for her safety out here.\\

In the apartment of her tiefling friend Trinia tells the companions her story. She was hired three months ago by seneschal Neolandus Kalepopolis to paint the king's portrait. On Pharast 24 she met his majesty for the first time, finding him 'sick' in his bed. Still, Trinia felt he wasn't all that ill and opened the shutters wide to allow light and air into the room. The king seemed to be suffering from a lingering cold and - if anything - a case of 'Werltschmerz'. The painter had also offered him some nettle tea, which is said to help against feeling tired and listless, a fact which Sjo confirms.\\

Trinia's presence, the tea and the fresh air seemed to agree with the king and he kept getting better every day, until he was no longer in his bed when she arrived, but already waiting for her in his seat, with a book in his lap and a fine bottle of wine by his side. Eodred II also turned out to be pleasant and charming company. His wife was less pleased with Trinia's presence. Her handful of visits usually ended in her nagging to the king that he should be more careful with his health and that he should stay beneath the covers until he had made a full recovery. She also felt that 'this painter' should stay away until then! So yes, Trinia was not a fan of Ileosa, although she might have expected such a reaction. The seneschal had freely admitted to her that her spunky personality and her youthful looks were another reason he had hired her, apart from her talent, which he fully acknowledged. He had hoped that she might return some of the king's lust for life.\\

By the end of Gozran Eodred's health had suddenly deteriorated. He got more feverish by the day, started coughing badly and developed strange skin sores. Trinia's time with him was limited to a couple of hours a day, and when the king got even worse, she was forbidden from seeing him at all. By this time the sores on his skin had turned to festering boils and he had almost gone blind. She even offered to finish the portrait without the king, but Kalepopolis had disagreed. He paid her 100 gold sails already and promised to contact her when Eodred got better. One week later the king was dead and the seneschal had disappeared without a trace.\\

Quint explains that Trinia's best chance is to prove her innocence. He offers to take her to the Citadel and convinces her that Field Marshal Cressida Kroft and high judge Zenobia Zenderholm are both very trustworthy people who will make sure she gets a fair trial. Trinia agrees and has Quint escort her to the Guard's headquarters. On the way they pass the lynch mob and although the hotheads do not recognize Trinia, who is wearing a disguise, she clings more tightly to the bard's arm until they clear the protesters.\\

Trinia repeats her version of the facts to Field Marshal Kroft, who feels it will be safest for the girl to remain here. Although no one enjoys the thought of being locked up for their own safety, Trinia realizes Kroft is right and agrees.\\

When they have concluded their business in the Citadel, the companions head for the old fishery, where Eric Brolan is still locked up after his assassination attempt during the opera. Quint explains that they will not press charges and will set him free, but insists that he contacts them when he comes across that Vudran snake in the grass, Selena, again. Eric is fully aware of how lucky he is and promises to do so.\\

