%!TEX root = ../crimson_throne_book_main.tex
% 2015-12-05
\section{8 Arodus 4708}

The next morning the heroes gather at the Twisted Door, the ancient set of double doors at the foot of the cliff, which serves as the entry to the Halflight Path. The doors are covered in strange runes in an unknown language. They also exhibit a strange optical phenomenon: the doors' edges appear perfectly straight, but an observer who follows an edge with his eyes somehow finds that was once an outside edge is now an inside edge. It is this effect that gives the door its name.\\

Derdra introduces the party to Tanack, a bearded man in a distinctive brown and gray uniform, the right breast of which bears a badge with a golden arch on a midnight blue background, representing the Twisted Door. He'll be taking them up as the first group. The guide hands the companions pendants that provide light in the darkness and beckons them to follow him down the tunnel.\\

Tanack explains that the complete height of the cliffs below Kaer Maga is riddled with huge complexes of natural and artificial tunnels and rooms, most of which have never been charted. Everyone in Kaer Maga maintains a healthy fear of these deeper levels, for they know that those who travel too often or too far beneath the surface return with tales of unspeakable horror -- if they return at all. From the earliest days of the city, its residents were under continual threat from the incursions of subterranean creatures that would occasionally make their way to the surface, which proved inadequately protected by its volunteer militia. Eventually, the city's residents decided that a small, specialized group could protect the city more effectively, and the first Duskwardens were commissioned. In the first years after their formation, the Duskwardens aggressively sought out and sealed as many entrances to the deeper levels as they could find, although they left a few select, easily-patrolled entrances accessible. Over the centuries that followed, the Duskwardens have patrolled the highest levels of the Undercity, maintaining the seals that prevent further attack from below and dispatching any creatures that manage to break through. So successful have they been in their mission that most residents of Kaer Maga are able to live their entire lives without ever encountering the denizens of the deeper Undercity. The Halflight Path is a carefully selected route through these tunnels. Tanack warns his clients not to stray from it.\\

The way up sneaks through endless tunnels. While the companions quickly start to feel the muscles in their legs burning, Tanack is light on his feet and often scouts ahead to make sure the dark holds no unwelcome surprises. Most side passages have been bricked up and sometimes Tanack warns the party to keep quiet, as he does not want to attract the attention of creatures who might finds these walls a mere inconvenience. At one point the companions hear a deep growl from behind one such wall, which is followed by a high-pitched death rattle. After two hours of traveling, the party reaches a round tunnel with gleaming walls that looks like glass. The tube is so smooth that only the sand spread out on the floor keeps travelers from slipping. At other points the tunnels turn into majestic hallways with ornate masonry and elaborate frescoes decorating the walls. The doors here are barred with locks and chains. When they are about halfway up, the companions pass three groups that are traveling down. Judging from the looks on those travelers' faces, Sjo concludes that going down must be more pleasant than making your way up, especially while wearing a full plate.\\

After five hours the heroes see sunlight at the end of the tunnel. Their hope that they have already reached their destination is quickly tempered by Tanack, who tells them that the tunnel emerges onto the cliff-face, becoming a ledge barely wide enough for a cart. The view is breath-taking, but one look over the edge at the sheer drop down is enough to keep the heroes hugging the wall. Tanack dances off ahead again to scout for danger, but when the party catches up to him, he is just standing on the ledge with his back towards them, as if he's waiting for them. He trembles as he turns around. Balian spots a line of blood coming from the corner of his mouth. Then he sees the three arrows protruding from the Duskwarden's chest. Tanack's legs buckle and he topples over the edge, plummeting down in the deep. Quint immediately reacts by throwing an {\itshape invisibility sphere} over the group, which is answered by a threatening howl from a fierce gnoll warrior who emerges from the rocks up ahead and storms down the path, into the invisible party. The hyena-headed humanoid is wielding a spiked whip. Two spear-bearing gnolls follow in his wake and a fourth warrior with a flail clears the bend as well. These creatures move fast; their speed has obviously been enhanced by magic. Sjo swings his heavy mace at the whipmaster, becoming visible in the act and revealing himself as a target to a gnoll archer who is hiding above the road. The healer gets two arrows in the chest. Puk tries to move into a favorable position by tumbling around the gnolls, but his enemies are not so easily fooled and the rogue suffers two attacks of opportunity: a lash from the whip and a stab from a spear. Balian orders Spyder to attack while Quint casts  {\itshape haste} on his friends and tries to take the archer out of the fight with a  {\itshape cacophonous call} , but the creature resists his magic. Balian jumps at the gnoll with the whip and brings down his greatsword with a force that could kill an auroch. The brute stumbles, but remains on his feet and counterattacks. Sjo pulls the arrows from his chest and heals himself with his most powerful spell,  {\itshape cure critical wounds} . In the meantime Puk slashes the wavering whipmaster in the back and brings him down. Spyder sinks his teeth into one of the spear wielding barbarians, who is now raging like a madman and stabbing at anything in reach. A final enemy reveals herself, as she appears from behind the archer and casts a spell that allows her to walk through the air. Then she suddenly blips out of plain sight, having turned invisible. Although Balian lacks the rage power of his opponents, his blade hits them with the force of a thunderstorm. But these gnolls fight smart, and focus their attacks on the ranger. As Sjo forces the enemy caster to reveal herself again with an  {\itshape invisibility purge} , Balian learns what it feels like to be a punching ball. A final arrow and a hit from the flail suffice to knock him out. Puk still dances around like a dervish warrior and takes out one of the raging spear fighters. The gnoll cleric calls upon the power of her damned god to {\itshape confuse} her opponents, but through sheer force of will, all companions and Derdra can bite off her foul magic. This seems to be the turning point of the battle. Puk gets knocked to the ground by the surviving barbarian's spear, but he keeps fighting from his prone position, barely clinging to life. Quint attempts another  {\itshape cacophonous call} on the archer, which  takes hold this time. Another enemy has been taken out of the fight. While Sjo pushes back the assailants with  a  {\itshape fireball} and heals his halfling friend, Puk kills off the last gnolls on the path. The air walking cleric notices the battle is lost and flies off, signaling her archer ally to get out as well. The party heals up and waits for the next group up to arrive. They explain what happened to the Duskwarden guide; who informs them that the Halflight Path houses many threats, but gnolls are not usually among them. This ambush must have been planned for the heroes in particular. The man takes the companions along with his group and after three more hours they arrive at the end of the last tunnel. They are now standing at the foot of the massive wall that does not only circle, but actually houses most of Kaer Maga. The hexagonal ring of eighty-foot high walls is imposing indeed. The residents have chiseled out doors and windows  at every height. There is no great gate into the city, but a giant breach in the northwest wall provides easy access through a district called the Warrens. The breach also allows the heroes to peek inside the hollow wall, which has been filled to the brim with layer upon layer of small buildings. Ropes, nets and wooden ladders weave their way between these houses like the rigging on a ship.\\

Derdra takes the party to an inn and leaves them there for the night, while she slinks off into the city to pick up the latest rumors on the Cinderlands and its Shoanti tribes. Quint urges his friends to join him to {\itshape The Flame That Binds} , the 'magic shoppe'. An amiable, but incredibly corpulent bloatmage welcomes them there, introducing himself as Carthagos. He offers to sell them magic components, scrolls and magical items. He is also willing to take interesting items off their hands, but having spent most of their fortune in Janderhoff, the companions have little left to barter with. 