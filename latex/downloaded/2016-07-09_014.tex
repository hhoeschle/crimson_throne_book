%!TEX root = ../crimson_throne_book_main.tex
% 2016-07-09
After having found out the basics on infernal contracts from Keppira d'Bear, the companions realize that they need to learn more about these devilish dealings if they want to stand a chance against Ileosa. The temple of Asmodeus is their best bet, but the place will be in state of alert after last night's raid, so getting more information from someone like priestess Tyresha will prove very tricky right now. The Acadamae probably houses a fair number of scholars who have a keen interest in the planes, but as the place is off limits to the public and closely guarded, it will prove even trickier than the temple of the Dark Prince. The Leroung library used to have a chamber dedicated to planar studies, but when the party was there two nights ago, they were told that imps had overrun the room and destroyed all tomes referring to Hell. So this is not a viable option either.\\

They call in professor Sirtane Leroung to ask her who in the city were regular users of the library's planar section. Unfortunately, she states that only Asmodeans and residents from the Acadamae frequented this particular reading room. So who else in Korvosa might know more about infernal contracts? The various devils themselves perhaps, but how would you go about getting them to talk? No, it feels like Korvosa will not offer this information freely. Balian suggests using the {\itshape teleport} scroll to go to another city to gather this knowledge: Janderhoff, Kaer Maga or even somewhere in Cheliax itself. Quint rejects Balian's idea, at least for the moment. The bard proposes to stake out the temple of Asmodeus to find out how things evolve there. He still sees the temple as his best option and feels it would be unwise to simply dismiss this possibility without making sure it truly is impossible. The bard also remembers Illrem Bromathan's plea for help to free his sister-in-law and nephew from their home, where they have been placed under house arrest. Since the party's first 'rescue mission' at the Leroung house didn't end so well, it seems like a good idea to stake out the Bromathan villa as well, before actually moving in to free the prisoners. Illrem provides the heroes with a sketch of the house and looks pleased that the party will heed his personal request, as the truly fears for the safety of his kin.\\

The companions decide to use the rest of the day for scouting. Sjo and Quint will watch the temple of Asmodeus, while Puk and Balian will keep their eyes on the Bromathan villa. Traveling to these two destinations will take them to the northwest and northeast of the city respectively. This means they have to part ways at the Gold Market, so they might as well have dinner there first in one of the restaurants. After all, it is no fun staking out a place on an empty stomach.\\

The Gold Market is unexpectedly booming with activity. A large crowd has gathered in front of a podium. Asking around, the companions find out that there will be a free theatrical performance, with free food and drink to boot. It looks like Ileosa is working hard on her public relations with another offering of bread and games. Deciding that they don't mind paying for a good meal, the companions find a great spot on the terrace of the exclusive King's Table restaurant, from where they have a good, albeit somewhat distant view on the stage. Only moments later the play begins.\\

---\\

\section{Korvosa always rises}

\section{Act 1}

\section{Scene 1}

The opening scene shows Eodred's harem: beautiful girls who keep the king occupied. After an exotic dance a smart-dressed man with a haughty attitude walks in and sends the girls away. It is the seneschal. The kings asks how his people are doing. Kalepopolis assures him that all is well and that his subjects are happy. The king does not need to worry, his seneschal has everything under control.\\

 {\itshape Balian recognizes the actor who plays Kalepopolis. It is Marco Ebhart, who was part of the}  \section{Scene 2}

Kalepopolis retreats to his chambers where he meets up with the Chelish ambassador Amprei. Amprei offers the seneschal a painting of a nude young woman, which Kalepopolis likes very much. He is especially fond of her boyish looks. Amprei claims it is a self-portrait of an up-and-coming new artist, named Trinia Sabor. Knowing the seneschal's love of the arts, he is willing to introduce the two in exchange for his help. He wants to have the king sign a new decree, forcing poor people who can't pay their taxes into slavery, as is the custom for halflings in Cheliax. Slavery will offer great business opportunities, as will the property market: slaves will be forced out of their homes and these empty houses will be up for grabs.\\

Kalepopolis likes the deviousness of Amprei's plans, but he seriously doubts Eodred will sign such a decree: the kind-hearted king loves his people too much. Amprei says he has already thought of that. He calls in a woman whom he introduces as Lady Andaisin, high priestess of Urgathoa. She can brew a potion that will make the king do anything they want. They will simply have one of his harem girls slip the potion to him. Kalepopolis is easily convinced, but says they will have to wait until after the Founding Festival tomorrow, for the can hardly send a drugged king into the city to greet its citizens.\\

\section{Scene 3}

During his ride through the city on Founding Day, the king sees how a young noblewoman from Cheliax gets thrown off her horse and he saves her from being trampled. He learns her name, Ileosa, and offers to take her to his castle to have her rest. A mother with a small ginger-haired child watches as the king and the noblewoman ride off. {\itshape``She is beautiful, isn't she, mother?}'' the boy says. {\itshape``She is, Tommy, she truly is}'', the mother replies.\\

 {\itshape There is something strangely familiar about Ileosa's voice and countenance. Studying her closer Puk sees that she looks and sounds exactly like Dame Nesia, the blood clone experiment who lived with the companions for a time before she exploded in a pool of blood. This woman must be Nesia's original!}  \section{Scene 4}

Kalepopolis is talking to Amprei and Andaisin. The seneschal is furious: Eodred has sent his harem away because he is infatuated with this new girl. The king has also decided to take a more active interest in ruling the city to impress his new conquest. Without the harem the conspirators won't be able to slip the king the potion. Amprei admits this is a great setback. Now they will have to find another way to force the poor out of their houses, so he can buy big building blocks cheaply. But Kalepopolis needn't worry, Andaisin has more tricks up her sleeve. After all, the best weapon against a woman with a gentle soul, is a woman with an evil one.\\

\section{Scene 5}

Eodred and Ileosa get married. The king promises to be a good husband to her and an even better king to his people. Naturally, he is concerned for his wife's safety and he gives her a personal bodyguard, Sabina Merrin and a handful of queen's guards, called the Gray Maidens.\\

\section{Act 2}

\section{Scene 1}

Kalepopolis has another meeting with Amprei and Andaisin. The seneschal is malcontent because he has lost a lot of influence over the last five years, since Eodred and Ileosa got married. The queen has made Eodred a better man and king, thwarting the conspirators' plans. Amprei whole-heartedly agrees, although he has the funds to buy up land in Korvosa's poorer quarters, prices are still too high. Andaisin says Urgathoa has the answer: the king will have to be killed. She takes out a potion bottle with a big white skull painted on it. But who will give it to the king? Kalepopolis smiles, he knows just the person. After all, it is high time the king adds his portrait to the gallery of monarchs.\\

\section{Scene 2}

Eodred and Ileosa are in their chambers, joking about what pose the king should assume on his portrait. Eodred wants Ileosa to be on the portrait with him, but she kindly refuses, saying tradition demands a state portrait of the monarch by himself. Then Kalepopolis interrupts, walking in with the artist, Trinia Sabor. Ileosa kindly greets her, but Trinia scoffs at her and demands to be alone with his majesty: there can be no distractions.\\

 {\itshape The woman playing Trinia looks familiar as well. Quint recognizes her as one of the Exemplary Execrables' former actresses. He even recalls her name, Greta. She was the young girl who portrayed the vessel for Donodarr's seed in Lucian Lycan's (Lick's) play}  \section{Scene 3}

Ileosa is talking to Kalepopolis. She is concerned for the king. Trinia has kept him locked up in his room for weeks now. Ileosa feels her husband should come out for some fresh air, the current situation is not healthy for him and it has also kept him from doing his royal duties. Kalepopolis wipes the queen's worries under the rug. Eodred is just getting older, he claims, and Korvosa does not need him to exert himself; Kalepopolis is perfectly capable of taking those heavy duties off the king's hands. Ileosa disagrees, she wants to take Eodred out for a walk on the terrace, but Kalepopolis blocks her way. {\itshape``Not yet,}'' he grunts, {\itshape``be patient, my queen, Trinia will finish her business soon enough!}''\\

After Ileosa leaves, Andaisin crawls out from behind the curtain. She asks Kalepopolis why the king still isn't dead. The seneschal says he might have misjudged Eodred's charm, he fears the silly painter has fallen in love with the king.\\

\section{Scene 4}

Trinia puts the final touches on Eodred's portrait. She walks up to the king and gives him a hug. Then she tries to kiss him, but he wards her off, saying those frivolous days are long behind him. He is happily married now. Trinia is obviously disappointed and asks the king if they can at least toast to the finished portrait. Eodred accepts. Trinia secretly pulls out the poison and pours it in Eodred's drink. Then they empty their glasses and Eodred drops to the floor. The king is dead.\\

\section{Scene 5}

Ileosa runs into the throne room and falls to her knees, crying because her husband is dead. Kalepopolis follows a few moments later, saying {\itshape``so it goes}''. People grow old and then they die. Then he sits down on the throne, shouting: {\itshape``The king is dead, long live the king!}'' As Ileosa looks up in despair, he smirks at her. If she is a good girl, he might find some room in his bed for her. {\itshape``Never}'', Ileosa screams in reply. The seneschal jumps to his feet again and draws his weapon. He claims that in ancient civilizations, wives joined deceased kings in the grave. Maybe it is time to revive these customs. He walks up to Ileosa threateningly, but Sabina Merrin shows up just in time to parry his blade. She wounds Kalepopolis badly and the man flees, swearing this is not over yet.\\

\section{Act 3}

\section{Scene 1}

Amprei and Andaisin are toasting the king's death, when a wounded Kalepopolis stumbles in on them. He tells them his plans to usurp the throne have failed: they underestimated Ileosa and her protector Merrin. Lady Andaisin calls this only a minor setback. All the conspirators have to do is prove to the people of Korvosa that Ileosa can't rule. Urgathoa will help them, she claims. She just needs some time to prepare a very devious plan, nothing short of a 'masterpiece'. Meanwhile she wants her allies to undermine the queen's authority. Kalepopolis cries he can't do anything for now, he's badly wounded! He will seek refuge with his Sable Company brothers to get well first. So it's up to Amprei, who promises to do what he can to incite the people to rebel against Ileosa.\\

\section{Scene 2}

Amprei is talking to Trinia, who is angry with Ileosa for being the true cause of the king's death. She calls Ileosa a Chelish wh*re. Amprei chuckles at this insult and tells Trinia to teach it to the filth in the streets.\\

\section{Scene 3}

Little Tommy, the kid with the ginger hair (now five years older) is walking with his mother, talking about how sad they are about the king dying. Suddenly they come across a group of people who are protesting against Ileosa (or the {\itshape``Chelish wh*re}'', as the call her); Amprei and Trinia are in the first ranks, shouting the hardest. Korvosan guards try to calm the demonstrators down, but Trinia draws a knife and skewers a guard. Sabina Merrin arrives on the scene, cradles the stabbed guard in her arms as he dies and then tells the protestors to return home. Amprei tries to stab her in the back while she is trying to defuse the situation. She manages to sidestep his vicious attack, knocks him out and then she draws her sword and holds it to Trinia's throat. She arrests the painter while she sends the rest of the rabble home. When everyone has left the scene, Little Tommy and his mother remain behind. Tommy asks his mother what a {\itshape``wh*re}'' is. She claims it is hard to explain, but whatever it is, she is sure Ileosa isn't one. {\itshape``What is she, then?}'' the boy wonders. {\itshape``She is the one who is going to take care of all of us, as queens do}'', his mother assures him.\\

\section{Scene 4}

Merrin brings Amprei and Trinia before the queen. Although she is still in mourning, Ileosa makes the right decision: she sends Amprei back to Cheliax, telling him that his title of Chelish ambassador is all that is keeping the axe from his neck. She tries to show some sympathy for Trinia, but the painter blames Ileosa for everything and refuses to show any remorse. With a heavy heart Ileosa condemns her to death.\\

\section{Scene 5}

As Trinia is being lead to the chopping block, Ileosa gives her another opportunity to confess her sins and repent, but again the girl refuses. Suddenly Blackjack appears and rescues Trinia. Merrin is frustrated, but Ileosa seems relieved: she is happy the gods helped her not to sink to Trinia's level and take a life. She is sure the gods will see justice done.\\

\section{Act 4}

\section{Scene 1}

Lady Andaisin is working in a dark laboratory, injecting a rat with a green fluid inside a large syringe. Suddenly Blackjack and Trinia walk in. Blackjack removes his mask. It is Kalepopolis. Andaisin wonders why he bothered saving Trinia, she is a failure! The priestess claims to be holding the key to their success in her hand, showing the rat to her visitors. Trinia mocks the priestess: {\itshape``You're going to take down the queen with rats? How will you do that?}'' {\itshape``Like this}'', Andaisin replies, as she lets go of the rat, who runs over to Trinia and bites the girl. Trinia is shocked and runs for the door. Kalepopolis intends to stop her, but Andaisin orders him to let her go. If the people don't kill her, the rat bite certainly will, she smirks.\\

 {\itshape Balian notices that the rat is not actually a real animal, but rather an illusion. He also briefly picks up a flash of the person casting the magical rat image: it is Rimando Lumenos, the gnome prankster who used to work in the Marble Dome for}  \section{Scene 2}

Little Tommy comes home to find his mother chase off three rats. He tells her he found something outside. He leads his mother out to a haystack where Trinia Sabor is in her death throes. Tommy's mother goes over to comfort Trina. With her dying breath the painter finally confesses: {\itshape``It was never Ileosa's fault, it was always mine.}'' Then she dies. Tommy's mother closes the dead woman's eyes and gets up with a cough.\\

\section{Scene 3}

Ileosa receives a number of plague doctors in her throne room. The physicians claim to have been sent by the Chelish government to help. Ileosa is grateful for their assistance, but after the doctors have left, Merrin tells the queen she does not trust them. The queen orders her personal bodyguard to keep a close eye on them.\\

\section{Scene 4}

Tommy brings his sick mother to the doctors' hospital. The beak-masked healers put her in a bed and give her medicine. They order Tommy to leave, as he will only get in the way. When the boy is gone, they laugh: no one knows their medicine will only make people worse. Sabina Merrin is standing behind a pillar, eaves-dropping. When one doctor leaves and only one remains behind, she knocks him out.\\

\section{Scene 5}

Sabina Merrin takes the doctor she captured to the queen. She forces the man to confess, but his acknowledgement does not contain a grain of remorse. He is happy to die for Urgathoa as his work in the city is done. Merrin gives him his wish and cuts him down. Ileosa is in shock. She feels guilty for allowing the physicians into the city and orders Merrin to put an end to their vile practices immediately.\\

\section{Act 5}

\section{Scene 1}

Sabina Merrin and her Gray Maidens invade the Hospital of the Blessed Maiden. They kill the doctors in the sick room and save the patients, including Tommy's mother, who is burning up with fever. She asks Merrin where her son is. The Gray Maiden commander tells her she will see him soon enough. She'll be safe now. Tommy's mother thanks Merrin, but feels that she will not survive. She is too sick. Then another Gray Maiden interrupts the conversation, saying she has found a way down to the underground temple of Urgathoa.\\

\section{Scene 2}

Kalepopolis and Lady Andaisin are in a state of panic, the Gray Maidens have discovered their hideout. Their sanctuary is under attack! Kalepopolis waves around papers which hold the recipe to the cure. He also has crates full of the medicine, which the doctors have been using to keep from getting infected themselves. Andaisin orders him to destroy the potions and papers, as she retreats deeper into the sanctuary, to the shrine of Urgathoa. There she will make her final stand.\\

After Andaisin leaves, Kalepopolis wants to burn the documents and smash the potion bottles, but Sabina Merrin arrives in time to stop him. Kalepopolis refuses to yield, shouting the throne should have been his. Merrin finishes him off by plunging her sword in his chest.\\

\section{Scene 3}

Lady Andaisin is praying to a golden statue of Urgathoa. Sabina Merrin walks in and the two of them engage in combat. During the fight, the golden statue animates and joins Andaisin in the fight. Merrin is hard pressed, but she wins out in the end.\\

\section{Scene 4}

Tommy runs to his mother's sickbed. She is really happy to see her son one more time and tells him to be strong, for her time is up. Then the queen herself walks in. She ignores her guards' words of caution and joins Tommy at his mother's side. She takes the woman's hand in hers and promises her she will save her. Merrin gets back and gives one of the potions to the queen, who immediately administers it to Tommy's mum. The woman suddenly regains her breath and feels better. Tommy is overjoyed. Ileosa asks Tommy if there is anything else she can do for him. He tells her to take care of everyone in the city, as queens do. {\itshape``I will}'', Ileosa promises.\\

\section{Scene 5}

Ileosa accepts the crown of Korvosa out of Little Tommy's hands, finding it only fitting to receive the royal headband from her own people. The boy is now perfectly dressed and Ileosa hails him for being everything a true Korvosan should be: brave, never-relenting and loyal to the city. In the rest of her speech she praises the citizens for surviving. The cure has turned the plague into a dark memory and brighter days are ahead. Ileosa swears to protect the people and serve the land of Korvosa. When the people on stage burst into cheers of {\itshape``Long live the queen!}'', their cries are taken up by the spectators. The play ends with Korvosa's new anthem: 'Gods save Korvosa, Gods save the Queen'.\\

---\\

After the play Quint makes his way backstage. He casts {\itshape glibness} and  {\itshape innocence} on himself and  weasels his way past the guards by calling out to his 'friend' Marco Ebhart, who is hanging around behind the podium. Marco does not know who wrote the play, explaining that the piece was provided by the magistrate of Expenditures, Syl Gar, who also hired the cast. He can tell Quint the name of the woman who played Ileosa, however, Elvira Campert, a student from the Acadamae. She is in the building behind him, which has been confiscated to act as the dressing rooms for the cast. Quint makes his way inside the house and finds Elvira in an upstairs bedroom, where she is removing her make-up. She has taken off her red Ileosa wig, \hyperref[fig:Noir-79828039]{ revealing her lush blond hair } . Now there is no denying it anymore, she is the spitting image of Dame Nesia! Quint introduces himself as a fellow actor and Elvira recognizes him from  {\itshape The Passion of Saint Alika} , which she claims to have seen. This is actually her first time on stage, she explains, and Quint showers her with compliments on her performance. She confirms that she is a student at the Acadamae, the Hall of Charm to be exact. As an arcanist specialized in enchantment, she has always felt a great affinity for the theatre, even though she never actively pursued this profession. She claims that she was asked to play the part of Ileosa by 'the castle'. When Quint pushes her to expound, she tells him that she has no real connections in the castle, apart from the new seneschal, Togomor, who is also the head of the transmutation department in the Acadamae. Although she did not study under him, the professor had a soft spot for her and tried to woo her in the past, without success. The man is powerful, which some girls might find attractive, but not Elvira. She is even repulsed by his appearance, which makes Quint smile. Togomor is not a good-looking man by any standards: old, balding and fat with a greasy skin. Not the type a pretty girl would fall for. Quint on the other hand is a handsome devil, and he feels that there is a spark between him and Elvira. He asks her if he can see her again. She says she would like that, but it is hard to get out of the Acadamae. She's not sure if there are going to be other performances of the play, none have been planned so far. So she asks Quint where she can find him, if she manages to get out. Since the bard can hardly tell Elvira he is staying at the resistance hide-out, he has to resort to quick thinking, claiming he is staying at  {\itshape Tenna's} , a luxurious bed and breakfast establishment in Citadel Crest. \\

\begin{figure}[h]
	\centering
	\includegraphics[width=0.4\textwidth]{images/Noir-79828039_mod.jpg}
	\caption{.: Noir :. by Charlie-Bowater}
	\label{fig:Noir-79828039}
\end{figure}

Next Greta - Trinia in the play - knocks on the door, asking Elvira if she is ready. Quint greets her and tells her he actually met her after Lick's (or rather Lucian Lycan's) play {\itshape The Hell Mother} , but she does not remember him. The bard is not surprised, recalling that Greta was 'high' when they met. Before he slips out, Quint manages to grab one of the cotton pads Elvira used to remove her make-up. This might be useful to scry on her in the future. Before he joins up with Sjo in the old courthouse, Quint swings by {\itshape Tenna's} to book a room in his name for several days. He leaves instructions at the desk to take a message for him if people come by when he is absent, which might happen often, he says. 