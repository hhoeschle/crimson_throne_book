%!TEX root = ../crimson_throne_book_main.tex
% 2014-05-03
After practice the companions pay Cressida Kroft a visit. Unlike a couple of weeks ago, the Field Marshal looks well rested. She's happy to see her friends and informs them that the situation in Korvosa has calmed down again. Quint tells her about their involvement in the opera and mentions their colleague Ali\"ena Fochs, whose 15-year old son was arrested during the riots. Kroft has no power over who remains in jail or who is set free, that is the job of the judges. She can write a letter to the prison warden, though, asking him to let her friends speak to the young prisoner.\\

With the letter the companions gain access to the dungeons of the Longacre building. They meet the warden, Duncan Bromathan. His grim face suits the underground environment in which he works, he looks nothing like his brother, lord Valdur Bromathan, who always seems to radiate with the light of Sarenrae. He takes his guests to see Alex Dervo, Ali\"ena's son, and allows them a few moments with the boy. He turns out to be a mean kid who has nothing but insults for the queen and his visitors. He does not want, nor does he deserve help.\\

With still a couple of hours of daylight left, Quint decides to follow up on a lead about his own background. In the archives of the old courthouse, he found notes on the key-lock killer's first victim, Leana Castel, a girl of the streets who was murdered in her room. Her baby survived the massacre and soldier Nimor Seelen was tasked to take the infant to the orphanage. Since the child had a scorpion birthmark on his shoulder just like Quint, it is clear that the bard was that baby. Seelen has retired from the guard and the companions find out where he lives.\\

When they present themselves at his house, an old lady opens the door. She has a baby on her arm and two more children are hanging from her skirts. Quint asks to see her husband. She shows the visitors in and points them to an old man by the hearth. He has a glazed look in his eyes and is gently rocking to and fro. Drool drips from his chin. "You can talk to him all you want," the old woman chuckles, "but I doubt you'll get any answers from him." Unlike her demented husband, Lieske Seelen is still quite vital and, having been raised properly, she invites her guests to take a seat. When Quint asks her about the baby that her husband took 18 years ago, she clearly remembers the child. She is proud of her grandchildren, she claims, showing off the baby on her arm, but the child her husband brought home so long ago, was absolutely handsome, a perfect little baby, if not for the creepy birthmark on its back. Quint confides that he was that baby and shows the scorpion on his own shoulder. Lieske gets quite emotional. She had wanted to keep the child, but her husband insisted on taking it to the orphanage. It wasn't until some years later, that she found out what Nimor really did, during one of their arguments about his alcohol and drug abuse. He blurted out that he hadn't taken the child to the orphanage as he was ordered, but had sold it to a crook in the streets in exchange for drugs. Lieske still feels guilty about it, but she's happy to see that Quint has turned out well.\\

Suddenly she remembers something. She starts looking through some drawers until she finds a white piece of cloth. It's an embroidered handkerchief with the letter "V" sewn in one corner. Lieske says Quint was wearing this expensive cloth as a diaper. She tells the bard to keep it, as it is the only thing that comes from his mother.\\

