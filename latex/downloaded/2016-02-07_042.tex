%!TEX root = ../crimson_throne_book_main.tex
% 2016-02-07
\section{15 Arodus 4708}

The companions return to Kroghut's court. Ark'ch, the warg rider, rejoins them on their way back. When the party reappears before Kroghut, the mighty warlord listens to their tale. Quint tells the story with gusto, making sure to underline Brugar's lack of willpower to resist the harpies and working up to a worthy finish in which he displays the gold bracers and fills his audience with awe for the beauty, power and value of these items. Such potent arm bands can only decorate the wrists of the most worthy, and who in these lands is more worthy than Kroghut the mighty? The warlord is pleased with the pinkskins' show of respect and rewards them with his clan's tokens: simple round disks of white wood with a crude rendering of the Broken Spine crest. He informs his guests that these medallions will allow them to get into the great city of Urgir, but he warns them not to dishonor his clan's name and also offers a word of caution. Pinkskins are not well liked in Urgir, even when wearing a token. They should do well to be respectful and even submissive to the orcs there if they want to stay out of trouble. Quint already realizes that this will prove to be a challenge.\\

Kroghut also invites the pinkskins to spend the night in his hall and travel to the city with one of his men tomorrow, Makpok the trader. The companions agree. Quint honors his host with some music, while Sjo practices his menacing grunts on the orc children who eye his gear too greedily.\\

\section{16 Arodus 4708}

The trip to Urgir takes about six hours. The heroes join a small caravan of five grain wagons under the supervision of Makpok. A dozen handlers and guards watch over the convoy. Makpok looks somewhere in his forties, quite a respectable age as far as orcs go. He turns out to be the friendliest orc Quint has ever met. The trader does not seem to mind dealing with pinkskins and proves to be a very good source of information on the party's destination. The Hold of Belkzen has no central government. Instead, the region is populated by dozens of orc tribes of various sizes, constantly warring amongst themselves or forming strategic alliances. Each tribe is ruled by a chieftain or warlord and some lucky clans have a savvy leader who has managed to gain hold of a fortress, like Kroghut of the Broken Spine. The mightiest of all warlords takes control of Urgir, the unofficial capital of the Hold of Belkzen. This formidable city is situated in the south of the Hold, along the central highway in the region, the Flood Road. This Riverbed lies dry for most of the year and only fills with water and mud for two months during the spring, when the snow on the Tusk Mountains melts.\\

Urgir itself is a mass of stone buildings and spires, dwarven monuments and deep warrens, held aloft by giant pillars of stone and iron deep beneath the earth. These supports suffer from a rust monster infestation, which leads to ever more frequent tremors and collapsing structures in the city as the columns are devoured by the metal-eating pests. Originally built as one of the Sky Citadels by the dwarves upon their emergence on the surface of the world and known as Koldukar, Urgir was conquered by orcs under the rule of the great warlord Belkzen in the Battle of Nine Stones, who renamed the city Urgir, meaning "first home" in his native tongue. Over the years, the masterful stone and metalwork of the city's dwarven builders has been defiled by its current residents, yet despite this defacing, Urgir remains a testament to its creators' talents and vision.\\

Since orcs find it difficult to maintain any stable government, the metropolitan Urgir proves hard to rule. Throughout the years, control of the city has shifted from one powerful warlord to another, and the ruling tribe has shifted more times than most orcs have the intelligence to count. The current ruler is Grask Uldeth, chief of the Empty Hand tribe, who has implemented a new, more accepting way of life. Envious of the human cities in neighboring and distant nations, the city's chief has opened Urgir up to foreign traders and travelers in the hopes of increasing the settlement's status in Golarion as a whole. His rule has proven quite successful, as Uldeth has held the city longer than any warlord in recent history.\\

The population of Urgir is almost completely comprised of orcs, with fewer but still numerous half-orcs nearly completing the city's demographics. In recent years, since Grask Uldeth opened the city to "pinkskins," a small number of humans, elves, and half-elves can be seen within Urgir's walls, primarily in the more trade-friendly sections of the city. Each of these non-orc inhabitants must carry and display a token from a tribal chief indicating that they have the protection and permission of that tribe to exist within Urgir's limits, and these tokens generally provide safety for the vulnerable minority races. That said, taunting and discrimination are prevalent from the orc natives, and while a pinkskin's safety may be guaranteed, they tend to stick mostly to themselves to avoid mistreatment from their hosts.\\

Using the famed dried riverbed of the Flood Road themselves, Makpok's caravan and traveling companions make good time. The closer they get to their goal, the more travelers they come across. Finally the heroes behold the wonderful city. Looming out of the rolling grasslands, a vast and white-walled city rises from the earth like a many-layered cake, tier upon tier of stone buildings and monuments, forming a mountain that glimmers in the afternoon sun. There is a long queue in front of the North Gate, awaiting permission to be allowed in. Makpok patiently gets in line and tells his fellow travelers that they will be on their own from now on. When they finally reach the gate, Makpok gains entry quickly, but the companions are hailed to the side for closer inspection. The orcs guarding the gate are dressed in breastplates bearing the image of a black hand. Although they lack the cleanliness of the Korvosan Guard, their gear looks in much finer shape than the arms and armor the companions saw on the Broken Spine warriors. Quint gets ready to address these orcs, when his gaze is drawn to a large shape breaking from the shadows.\\

An ettin, nearly as tall as the fortifications themselves, walks up, towering over the orc guards. The creature is naked, safe for a loincloth, and has vaguely porcupine features, seeming a grotesque blending of orc and giant. More disturbing than its stature, however, are its two heads -- one (apparently the dominant) stares at the companions with rheumy eyes, while the other stares off into  space and simply drools. Around the neck of the first head swings a pendant bearing the black hand symbol. The beast carries two spiked clubs, each thicker than a man's chest.\\

{\itshape``Whassup?}'' it gurgles.\\

{\itshape``Pinkskins come to make trouble, Wargus}'', one of the orcs calls.\\

The ettin towers over the heroes: {\itshape``Token?}'' it grunts.\\

Quint quickly shows his Broken Spine pendant. {\itshape``Broke Neck}'', the giant observes. {\itshape``What yer bizniz here?}''\\

{\itshape``We come to trade. We are adventurers, looking to sell some interesting loot to our orc brothers}'', Quint smiles confidently. {\itshape``We hear that Urgir is not for the faint of heart, and since we are nothing of the sort, we feel like we need to honor your great city with a visit. We also have some nice things for sale, so it will definitely be worth some orc's while, I can assure you.}''\\

{\itshape``Big words, pinkskinner}'', the ettin growls. {\itshape``But good token, you pass!}'' At that the mountain of meat moves back into the shadows of the gatehouse and the orcs step aside, clearing the way for the companions to enter Urgir.\\

This entry to my journal drew greatly upon some fine sources: the Urgir page on the Golariopedia (pathfinderwikia) and 