%!TEX root = ../crimson_throne_book_main.tex
% 2014-07-24
By one o'clock the companions head to Castle Korvosa. They notice a lot of Gray Maiden guards, who escort them to the grand salon, more specifically to the balustrade that looks down on the stage hall where Ileosa's coronation celebration took place. The four young men are now on the balcony where the Gray Maidens presented their large numbers to the queen's guests the day before yesterday. There is a select group of witnesses in the room with them: Field Marshal Cressida Kroft and Commander Marcus Thalassinus Endrin of the Sable Company, Lords Mercival Jeggare, Valdur Bromathan IV, Glorio Arkona and Toff Ornelos. Lady Eliasia Leroung and Archbanker Darb Tuttle of Abadar complete the party. Lady Zenderholm is in the stage hall below, accompanied by two clerks, who have to meticulously record what will be said.\\

When the clock strikes one Ileosa enters the room below and sits down on a high-backed chair on the stage: "Honorable guests, it pains me to tell you that the rumors concerning my late husband's death ... have proven to be true. We owe the capture of his murderer to our beloved heroes", the queen says, pointing to Balian and his friends. "The girl has already confessed, but law dictates that she repeats her confession. Normally one judge and two witnesses would suffice, but I felt it wise to invite all of you here to bear witness to the horrible truth. Judge Zenderholm will be conducting today's questioning. She will use a {\itshape zone of truth} to ascertain herself and all of us that no lies are spoken." At Ileosa's command Sabina Merrin leads Trinia Sabor into the room. The young painter is not shackled and willingly walks onto the stage. Judge Zenderholm calls on the power of Abadar to install the {\itshape zone of truth} around Trinia and starts her brief interrogation: "Trinia Sabor, you stand accused of killing King Eodred II. Do you confess to having murdered the king?"\\

Trinia answers without showing any emotion: "I do, I killed the king."\\

The witnesses on the balcony all sigh with indignation. The companions are taken aback by Trinia's icy response, but they cannot sense that she is telling anything but the truth.\\

Zenobia Zenderholm continues: "Can you tell us how you murdered the king?"\\

"I poisoned him", Trinia plainly states, "which made him ill. But he wasn't suffering from any illness or disease, since it was poison, so curative magic did not work."\\

"And where did you get this poison?"\\

"Korvosa's black market is much bigger than you can ever hope to fight as a judge. Everything is for sale, if you know where to look. Of course, smart sellers know how to remain anonymous, so I bought the poison illegally, but I cannot tell you who I bought it from."\\

"And where did you buy it, then?"\\

"In Eel's End, in Old Korvosa."\\

Judge Zenderholm then asks a very important question: "Trinia Sabor, which motive did you have for murdering the king?"\\

Trinia remains as unemotional as before: "I wanted to become court painter, permanently, I mean, not just for one job. I had offered myself, my body to Eodred to get my wish granted. Everyone knows that his majesty has always had a weak spot for pretty girls, but he refused me. He said he loved his queen too much and did not want to betray her. So I killed him ... because I don't like being rejected."\\

Another shock goes through the audience on the balcony, not only for Trinia's words, but also for the detachment with which she speaks them. Meanwhile judge Zenderholm concludes: "That is horrible, Miss Sabor. May the gods have mercy on your soul. The punishment for regicide is death by beheading!"\\

Sjo notices that Trinia's icy facade breaks for a split second. He reads fear in her eyes and sees her hands tremble ever so slightly. Quint studies Sabina Merrin's reaction, but notices no schmuck smile or contented reaction at the painter's conviction.\\

Three Gray Maidens lead Trinia Sabor away, while the queen, who is obviously affected by the ordeal, excuses herself and calls an end to the meeting. As they are escorted out, Quint asks a Gray Maiden to see Trinia Sabor or Sabina Merrin, but the guard replies that her only task is to take the guests out of the castle. The bard notices she speaks in a distinct Chelish accent. As he gives air to his discontent, Lord Arkona asks him why he is such a disbeliever. Quint explains that today's performance was not in keeping with Trinia's character and he refuses to believe that such a trite excuse as "anger at being rejected" led to a kingdom in chaos. Lord Arkona agrees that Trinia's motives were quite trivial, but if true, they could have led to all that happened, trivial or not. One thing is clear though, this motive begins and ends with Trinia and hides no darker plot that threatens the monarchy, which is probably good news. Still, Quint is not convinced. He also wonders when the execution will take place. Arkona figures it won't take long, probably as early as tomorrow.\\

The companions realize they have an enormous problem on their hands, but there is little to nothing they can do. Trinia confessed and will be executed for it, unless they can prove her innocence. They go by the temple of Sarenrae: high priest Ezekiel Sollux examined the king when he was ill, maybe he can shed some light on the case.\\

The priest of the Dawnflower has already heard about Trinia's confession and even knows that her execution is set for tomorrow, as the companions already feared. He explains that he was only called in when his majesty was getting worse, maybe two weeks before his demise. Until then Archbanker Darb Tuttle had been treating the king. Nothing in the king's symptoms pointed at poison, Ezekiel claims, so the news that Trinia poisoned the late monarch came as a surprise, but it could have explained why healing magic did not help. As far as he knew, the king was suffering from an unknown, but incurable disease. Priests do have spells that enhance the natural process of getting better, he clarifies. Unfortunately there are diseases that no one recovers from, that are simply lethal. Even the strongest magic can do nothing then. He figured that this was the case with Eodred's mysterious illness. He has never heard of a poison that replicates the symptoms of leprosy, but he doesn't rule out the possibility that such a poison exists.\\

As there is nothing else to be done, the companions return home. Sjo makes good on his own plans and spends a few hours in Larella Semyr's lovely company at the shrine of Shelyn. His relationship with the priestess grows ever closer.\\

