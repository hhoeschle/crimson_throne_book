%!TEX root = ../crimson_throne_book_main.tex
% 2014-05-17
\section{10 Sarenith 4708}

The next morning the companions head to the orphanage where the other little saved lambs are staying. They ask them to make drawings of the queen and her new heroes. These simple childish tokens of affection will show her majesty that her people love her.\\

Then they make arrangements for their own present. They buy a nice jewel of a pseudodragon at the Golden Market and pay a visit to Tepest Geezlebottle, the gnome professor and headmaster of Theumanexus College, Korvosa's smaller school of wizardry. They spend their last cash to have him enchant the jewel as a special version of a {\itshape bird feather token} . The jewel replaces the feather and the token will create a pseudodragon instead of a bird to deliver a message directly to them. They will place this shiny object in the center of a clay table with impressions of their hands around it and a small plaque with the words:  {\itshape Always ready to lend a helping hand} . At four o'clock the complete cast gathers in the Marbledome. Everyone takes a seat as Lord Mercival Jeggare climbs on stage:\\

"Good friends. I've learned that you did an outstanding job this past week. I'd like to congratulate you for all your efforts. I can rightly say I'm proud of you.\\

Of course, I don't have to remind you of how important tonight is. We'll be playing to an audience that even this great theater has never seen. All the greats of Korvosa will be present. Even her majesty will be gracing these halls with her presence.\\

You know that the recent weeks have been hard on our city. After the regrettable loss of our King, Eodred II, we've lived through troubled weeks with heavy riots in the streets. Many have lost a loved one in this meaningless protest. But Korvosa would not be Korvosa if we couldn't find the strength to rise above this misery. Fortunately peace has once again returned to our streets, albeit in a rather fragile form.\\

Your performance tonight serves to strengthen that peace. Alika was one of our most loyal and cherished citizens, who gave her life to protect Korvosa. An opera honoring her shining example will rekindle our sense of unity and repair the heart for Korvosa in those who have lost faith. In this respect your job here tonight is much bigger than an artistic accomplishment. You are the ointment that will heal our city.\\

I'm not asking you to give your lives for Korvosa, as Saint Alika did. I am asking you to put on a show as if your lives depend on it. Perform like you've never performed before! Make this show the highlight of our time. Make sure that, for the rest of their lives, our audience will claim that THEY were here to see it. My heart swells when I see all of you in front of me, full of expectation, full of fire. Tonight you make your mark! Make every citizen honored to be a Korvosan, to be part of this wonderful city with its glorious history, her formidable heroes and her first-rate artists. My friends, make Korvosa proud!"\\

Jeggare's speech is well received and bolsters the cast's resolve to shine. With less than four hours to go, everyone rushes off to the dressing rooms to put on their costumes and make-up. There is a healthy tension in the air that focuses the mind. Singers warm their vocal cords, actors go over their roles one last time or run through the combat moves again. Quint dedicates his time to motivating some of the more nervous cast members to forget about their jitters and enjoy the moment. Eric Brolan, who plays Death in the final scene, has a bad case of stage freight, but Quint can talk him into a calmer state of mind.\\

Before long the first spectators arrive. Tension starts to rise behind the curtains. The buzz from spectators arriving grows ever louder and adds to the excitement of the actors and musicians. Suddenly a loud applause erupts as people shout: "Long live the queen! Long live Korvosa!" Her majesty has arrived.\\

A few moments later it is time to start. A new round of applause fills the hall as the orchestra starts playing. Puk is in his element tonight and beats the drum with unseen passion and feeling. Then the curtains open. Varric Bedan, who has been struggling all week, manages to keep his voice clear tonight when bringing Waydon Endrin's speech to the settlers. His past as a soldier has obviously taught him to keep his act together in tense situations. His portrayal of Korvosa's first Field Marshal is well received, especially when Arianna Evenland joins him as his wife captain Keyra Palin, another hero of old. But mostly anticipated is Saint Alika herself. When she is singled out by Waydon Endrin as the first child to have been conceived and born in the settlement, some people in the audience already start whistling with anticipation. Palastus, who's watching closely from the sidelines, smiles happily, which is a rare sight.\\

As most actors leave the stage, Alika stays behind with her father and the Varisian trader played by Quint. Alika's voice is beautiful and full of emotion as her father presents her to her future husband. The orchestra continues playing while the curtains close and Dargo the Hump quickly changes the scenery by pulling some ropes.\\

The second he's ready, the curtains swing wide again for act two. Alika wanders into the wilds, crying out her despair and begging the gods for help. At the other side of the stage Dario Darnas appears in the guise of the good-looking Chelish soldier. Despite his oafish demeanor, he is a crowd's favorite and proves that he deserves the part of Kallan. When he meets Alika, there is a tangible chemistry between the two. Aisha shows no sign of the disdain she normally bears for the actor and sparks fly. The audience loves it. People sigh at the forbidden love scene and hold their breaths while Alika receives her first vision of the Shoanti assassin. The companions witness for the first time how incredibly impressive illusions look in the dark. When Marco Ebhart tries to assassinate the two lovers the audience roars in outrage, but when he is killed in the following fight, they cheer.\\

The set changes back to the fort for the third act. Quint is in his element in the scene where he has to mock Alika and her lover. Being a comedian by training and having a sound dose of dislike for Dario Darnas in real life, making fun of his rival comes naturally to the bard. Another star of this act are the special effects: Dargo and Lumenos summon a livid vision for Alika and their storm rages with a power seldom seen on stage. Again the audience goes wild as the curtains close for the break. The opera is well on its way of becoming a hit!\\

Behind the curtains Lord Mercival Jeggare can't hide his excitement. He pats Quint on the shoulder while raves about the marvelous show. His face lights up as he speaks. Then Balian notices an unexpected guest behind the scenes. Field Marshal Cressida Kroft approaches fast, with a figure in a grey cloak by her side. The look on her face tells him she is not here to praise the actors. On the contrary, her eyes radiate anxiety. "We have to stop the show immediately," she pants, "Aisha Leroung is about to get murdered!"\\

"Slow down, Field Marshall Kroft, what is going on?" Mercival Jeggare wonders.\\

The cloaked figure removes her hood and the companions recognize Keppira d'Bear, high priestess of Pharasma. "I have just received word that Aisha is going to be killed. You must abort the show, now!" she answers. "Allow me to explain, you may remember that a young girl in the village of Sirathu recently started displaying the same talents as Saint Alika, the gift of Pharasma to see the future. I sent my right-hand man Jasan to the town to determine whether the girl really has the power of foresight.\\

Just now, during the show, Jasan magically sent me this message: {\itshape New seer of Pharasma predicts doom. Alika will be murdered in the last scene by someone on stage to throw the city back into chaos} . If anyone realizes that Pharasma's visions have to be taken seriously, it has to be you. I'm afraid we have to call off the show!" Mercival Jeggare reacts furiously: "We can't stop the opera! We're a hit! We're finally uniting a divided city. Calling off the show now is as detriment to Korvosa as the predicted murder itself! If we give in to the powers that want to tear us apart, we will never rise from the chaos. The show must go on! We have Korvosa's heroes at our side. These young men will watch over Aisha and protect her if the need arises. How sure can we really be about your newly found oracle? Your right-hand man has not confirmed her gift yet, has he? Suppose he is wrong, then we will be the factor that sows new chaos. No, we're here to bring unity. The show will go on! And not only that, if we want to know who's trying to undermine the city, we have to catch him in the act. If not, we'll never find him. If we call off the opera now, we will never find out who threatens the peace. If you're serious about taking out the enemy, you have to know him first."\\

Kroft disagrees, but Jeggares stands by his decision and the companions take his side in the discussion. They will keep an eye out for trouble and interfere if something happens. "But", Jeggare insists, " if you have to act, try to do it in such a way that the opera can continue. Incorporate it into the story, just go with the flow. There's too much at stake." The Lord also advises Cressida Kroft to head to the queen's box and stand watch over her. "The heroes will hold the 'fort' down here", he assures her.\\

The second part of the opera feels totally different to the companions. The audience remains enthusiastic at Waydon Endrin's struggle between despair for losing his wife and hope that she may have survived as Alika has foreseen. Everyone screams with joy when the field marshal is finally reunited with his spouse. Meanwhile Quint and his friends are trying to figure out who the killer might be. The last scene has over thirty people playing dead on stage. Everyone of them could be it. But their prime suspects are Eric Brolan, the man who plays Death, and Ali\"ena Fochs, the angel who takes Alika to heaven. As Quint and Balian are both appearing in the current scene, they don't have time to look for those two actors, though.\\

The fourth act in the Shoanti camp also elicits a lot of reactions from the crowd. Some people boo the historic enemies of the city, but most spectators get carried away by the energy of the scene. Sjo makes an excellent debut as the Shoanti leader Garak and shows no signs of being distracted. Although the fourth act is the shortest one, being comprised of only one long song, it gives Quint and Balian a few minutes to look around. Balian spots Eric Brolan by the toilets, talking to some of the extra's. He is already cloaked in his death shroud, making it hard for the ranger to see if he's hiding anything underneath his clothes. There is no immediate sign of Ali\"ena Fochs.\\

Then it is time for the final, possibly fatal act. The companions are focused on every diversion from the original script, but with so many weapon-wielding actors about, it is hard to tell who's a real threat. Balian misses his one second of fame completely, blurting out "Flame!" instead of "Fire!". Meanwhile Shoanti warriors and Chelish soldiers clash against the flaming background of a fort on fire. Aisha is constantly in the thick of things, but she remains unharmed until Sjo has to 'kill' her in the show. The Shoanti makes sure to land at Aisha's side when he is defeated himself in combat, actually landing in the spot where Dario is supposed to kneel besides his slain love.\\

That is when the Grim Reaper arrives on stage. Balian and Sjo are among the many fallen on stage, while Quint, who had to flee the scene, is watching closely from the wings. Suddenly he spots a sword under the Reaper's death shroud. Puk sees it as well and readies himself to interfere. Quint returns to the stage, improvising a song that he has come back to fight off Death, rudely interrupting the Chelish diva Auralia Lazanne, who is singing her beautiful aria in the orchestra box. Fortunately the band members continue playing and to the unknowing spectator's eye, this all looks like part of the show. Quint quickly crosses the few steps to Brolan's side, just in time to intercept the short sword that he draws from under his cloak and knock it from his grasp, although the bard has to use his new spell {\itshape timely inspiration} to succeed. At the same time Puk jumps on his biggest drum, using it as a trampoline to leap on stage and stab the foul assassin in the back. Balian, who made sure to 'die' close to Alika as well, also rises and swipes his greatsword at Eric Brolan while the man draws another short sword. Sjo, who fell behind Alika, whispers the words of a  {\itshape shield of faith} to protect the lead singer, but that does not prevent her from being wounded by a foul sword trust. Still, not being surprised by the attack, she avoids a ton of nasty sneak damage, which probably saves her life. Sjo urges the girl to stay calm when he notices fear slipping into her eyes. Quint, Balian and Puk press the attack, trying relatively successfully to blend their fighting into the show. Balian hammers the flat side of his blade against Brolan's head as Puk slips under the man's defenses again and delivers another bloody stab wound. While Sjo heals Aisha, further soothing her fears, the attacker goes down. Puk quickly regains his seat in the orchestra box and Quint nods to Auralia Lazanne, signaling her to pick up her aria again, while he and Balian retreat to allow the descending angel to take over and end the opera as intended. Although she is quite dazed by the experience, Aisha manages to pull herself together enough to play out Alika's part gracefully. When the show is over the Marbledome booms with applause. While their fellow actors were surprised to see the strange course of action in the end, no one in the audience noticed anything out of the ordinary. Palastus shoots the companions an angry look before he steps on stage and invites his complete cast to bathe in the cheers of the spectators. Quint even keeps his promise to the gnome illusionist, making the director fart with {\itshape ghost sound} as he bows, although the loud crack gets completely lost in the crowd's cheers. When the curtains finally close, Palastus' happy face switches to thunder. "What in all the hells was that?" he shouts. 