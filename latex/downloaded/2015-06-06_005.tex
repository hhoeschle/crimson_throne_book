%!TEX root = ../crimson_throne_book_main.tex
% 2015-06-06
"There's something up there." Balian's senses alert him to the ticking of rocks up the mountain side. When he looks up he sees loose stones starting to shift and slide, quickly gathering speed and volume. The ranger jumps back, getting himself and his steed Webb out of the sudden avalanche that floods the pass. Just in time; if he had reacted a heartbeat later, he would have been buried by the rubble.\\

Glancing up the mountain flank, Puk realizes that the landslide was no coincidence. Three large shapes are moving against the rocks. Their dark gray skin makes them blend in with their surroundings, explaining why the companions did not notice them before: stone giants! Sjo calls upon his flame-wielding gift and hurls a fireball on the ambushers, but the giants emerge from the blaze hardly scorched. Two of the attackers jump down, swinging their huge clubs, while the third starts hurling boulders on the travelers. The heroes manage a handful of nicks and cuts, but it will take more than a few hits to take down these brutes. The giants on the other hand pack a very mean punch, one of them pummels Sjo to the ground, while the other one nearly bashes Puk's skull to pieces with a mighty critical strike. The halfling bites back the darkness that threatens to drown him and gulps down a healing potion, while Sjo barely manages to heal himself with his magic. Quint has taken a few steps back down the path and casts {\itshape mirror image} on himself. He then taunts the giants with his song of  {\itshape satire} , luring one of the ambushers to his position. The large female tries to smash him with her greatclub, but she hits an image instead. Meanwhile Balian finally takes down the first opponent. Puk and Sjo rejoin the fray and focus their attacks on the female, while the rock-throwing giant - clearly upset by the death of his brother - storms down the mountain as well. Thanks to the distracting barbs of Quint's  {\itshape satire} the stone giants start missing their attacks. Bleeding from many wounds, the gray female sags to her feet, finally defeated by countless cuts. Her ally joins her in death a few moments later. The heroes quickly heal up and discover a treasure bag on the female stone giant. It holds a cave bear pelt, some gems, a gold nugget and a giant-sized horn with intricate carvings. Quint figures it would make a nice addition to the Jeggare Museum in Korvosa. The companions continue their journey and reach the green wilderness of Bloodsworn Vale as the sun glides behind the highest peaks. There is not much sunlight left, so they'll have to find a place to set up camp for the night soon, especially since dark clouds are gathering in the sky, heralding a summer storm. Checking the map that professor Sirtane Leroung provided, Sjo concludes that there should be an old fortress down the road: Fort Thorn. The stronghold supposedly dates back to a time when the road to Nimrathas was still used. The healer wonders what will be left of the structure after a century of disuse. The ancient road through the forest has worn down to barely more than an animal path and does not easily allow passage to four travelers leading four stout mounts. By the time the light dies, Balian spots the old fort. The main building, which was made of stone, seems mostly intact, but the rest of the fortress lies in ruins. The wooden palisade has been reduces to crooked stumps of wood that are overgrown with bushes. There is just enough of the old gatehouse left to identify it as the erstwhile entrance. Fort Thorn's central keep was once a massive two-story building, but now it looks more like a haunted house. The eastern flank is heavily scorched and overgrown with the valley's trademark bramble: bloodroses. There is no sign of blood cap mushrooms, though. Balian figures the toadstools require a more humid environment. As the first flashes of lightning illuminate the evening sky, the companions make an uncanny discovery: the dead body of a small, winged humanoid has been nailed to the keep's front door with three arrows. The creature was obviously left there as a warning. Quint identifies the unfortunate being as a pixie, tiny feyfolk who are known for their elusive nature and whimsical character. Balian examines the arrows and finds out that the tips are made of cold iron, the only metal that can hurt these small fairies. Sjo claims the pixie has been dead for two days.\\

The keep itself looks sturdy enough to provide shelter for the night and the coming storm, but the heroes want to make sure that the rest of the fortress is safe before they make camp. The eastern half of Fort Thorn was completely laid to waste by fire. Wild bushes have reclaimed this speck of civilization, making it hard to even figure out where the various buildings used to stand. One ruin still rises: the base of a stone tower. This building used to house a small smithy. It looks like someone recently cleared the inside of weeds and bushes and made use of the anvil to forge weapons out of cold iron, as some grains of blue-gray ore are still scattered over the floor. Balian examines the place carefully and deduces that it must have been about two weeks since this smithy was cleared and used. Summer rain has worn away most tracks, but the ranger finds one clear footprint in the sand in front of the tower: an unshoed horse hoof. Beyond the tower, Sjo rummages through another growth of bloodroses and notices that they actually cover a statue representing Erastil, the god of wilderness and the hunt. Somehow this wooden sculpture survived the fire that burned down the rest of the building a long time ago.\\

With the rest of the fortress secure, the companions get comfortable in the old keep, while a thunderstorm erupts outside. They are happy to find out that the ground floor of this building remains dry, even after a century of decay.\\

