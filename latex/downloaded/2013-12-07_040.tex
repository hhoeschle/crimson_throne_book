%!TEX root = ../crimson_throne_book_main.tex
% 2013-12-07
The young friends return to Citadel Volshyenek. On their way over there, they pass a mad prophet on a street corner who has gathered some unlookers. The wild-haired old man preaches that " {\itshape the eye of Groetus has turned from the Boneyard of Pharasma to look upon Korvosa} ", claiming that "  {\itshape Korvosa's darkest hour is still to come! It will spell doom for all! Time to make peace with your loved ones, before it's too late!} " Sjo and Quint's witty remarks that they are already at peace with everyone and that if there will ever be an end of days, it is inherently drawing closer every day, makes the raving lunatic lost for words. When the companions arrive in the Citadel, they come across some soldiers practicing in the courtyard. One of them, a woman in a helmet, faces a tall, cocky man, who taps his sword on his shield while joking: " {\itshape So, you all expect me to hit a woman? Save yourself the trouble, honey, lay down and allow me to plunge my sword into your body.} " The woman suffers this jests undauntedly and return the favor: "  {\itshape From what I hear that sword of yours is not much more than a butter knife, Barvos. Come to mama, then we'll see who really has the balls in the family!} " The male guard dislikes being made fun of himself and leaps forward, lunging wildly at his female colleague, who deftly steps out of his reach and kicks him in the behind as he surges past her. The man returns more cautiously now, trying to force back the woman with more measured strikes. She catches each one of them with her sword or shield. The mere sound of the blows suggests that the man is putting a lot of his strength in them. Then the woman takes the offensive: she is quicker and more precise and now the man has to give way to her. When frustration takes over, he wildly bashes his shield against hers, putting all his body weight into the punch. His attack throws back the female guard a couple of feet and breaks her advancement. The two adversaries now start circling each other, using more predictable tactics trying to hit one another. After a few minutes the female guard seems to be getting tired, as her shield sags a few inches. Her opponent seizes the opportunity and leaps at her, but her tiredness was just a ruse. She dives under his sword and hooks her blade behind his foot, tripping him. As the male guard bites the dust, the woman kicks his weapon from his grip: " {\itshape Those balls of yours need to grow some more, Barvos} ", she jokes. Then the woman notices the companions. She removes her helmet to reveal a sea of lush red hair. As she faces the young heroes, they recognize her as Amarice, their former fellow lamb. She is madly enthusiastic about seeing her old friends and greets them warmly with big hugs and a broad smile of joy. Since she is still working at the moment, she agrees to meet the companions later this evening at the Travelling Man in Old Korvosa.\\

In Kroft's office the young heroes fill in the field marshal on Verik Vancaskerin's new operation. They explain that they discovered little political motivation in Verik and his men and wonder if anyone is forcing him to cooperate. Cressida clarifies that Verik is unmarried and has no ties in the city, since he immigrated here from Riddleport. So she does not believe that someone is threatening his loved ones to make him do something against his will, as he doesn't seem to have close relationships in town. There are hundreds of soldiers in the guard who would make much better targets for such a scheme than Vancaskerin. Sjo suspects they will have to search for whoever is 'sponsoring' Vancaskerin in Korvosa's higher circles, but cannot provide a ready answer to the question what drove the sergeant to quit his post.\\

Upon hearing that Vancaskerin is more worried about getting food to the people than spreading lies about the queen, the field marshal stresses once more that the companions have to do their best to take him alive, as her deserted sergeant might not be a bad man as such, just misguided. Still, his treason is a glowing ember of rebellion among other soldiers who contemplate quitting the Guard for personal reasons, most of them to protect their families. So the deserters have to be taken care of as soon as possible. The normal punishment for desertion in times of military need is death. Kroft says she is prepared to look at Verik's case personally to see if he merits a less severe punishment, but he will have to be punished one way or another. She also advises the companions to transport Verik and his men to the Longacre Building if they are taken alive. Longacre houses the city's courthouse and prison, and it is much closer to Vancaskerin's new headquarters than the Guard's Citadel.\\

