%!TEX root = ../crimson_throne_book_main.tex
% 2015-12-19
\section{10 Arodus 4708}

As the journey continues, the travelers come across a gruesome sight. Three Shoanti heads have been shoved on a pike, with two crossbow bolts sticking from their eyes. Derdra says that this is the work of the Cinderlander, a lone wolf who has devoted his life to hunting and killing tribesmen. Some think he is a ghost, but Derdra has heard more plausible tales of a man from the Korvosan lowlands whose family was butchered by a Shoanti war party and who is out for a lifetime of revenge. Sjo gets goosebumps when she tells this story, for it matches the tragedy he recently learned about his mother. The warriors that executed this raid came from her tribe. Being too young to join them, she followed them instead, with her best friend Shaoban. The two youngsters were surprised by a single survivor, who killed Shaoban and raped the girl, which would make this Cinderlander Sjo's father!\\

After many hours the heroes reach the Kallow Hills. Derdra finds mounds of stone with animal skulls on top: they have entered the domain of the Skoan-Quah. The guide describes how this clan fulfills a special role in Shoanti society. The Skull clan is devoted to the memory and protection of the dead. Its warriors cover themselves in a mixture of mud and white ashes, giving them the color of smoke. This is rumored to protect them from the undead, whom they have to destroy out of tradition and duty. The Skull shamans travel between all the other tribes to bless their dead and carry the most honorable deceased to these hills, to ensure a safe passage to the eternal hunting grounds. This makes the Kallow Hills one of the greatest burial sites of the Shoanti.\\

Half a mile into the hills, the companions notice four warriors covered in grayish white paint. Quint waves at them, showing no sign of threat, and hails them over. They greet the visitors with caution: {\itshape``Sharatok tshamek''} -- be received, strangers. Sjo pulls out the amulet Thousand Bones carved for them back in Korvosa and says in Shoanti that they are here to talk to the wise shaman. The warriors relax and lead the party to their camp.\\

Dozens of tents have been set up around a central totem. This camp must house at least a hundred people. Derdra explains that the Shoanti are divided in seven great clans, but each clan in turn is divided in smaller tribes, as the Cinderlands are too inhospitable to support massive groups of people in the same place. The tribesmen regard the visitors with great curiosity, but when Thousand Bones steps out of the central tent he immediately recognizes the companions and gives them a heart-warming welcome. He also introduces them to his chief, jothka One-Life, and Ashdancer, another shaman. Balian also notices a large wingless dragonlike creatures lurking between the tents. Suddenly a cry of happiness resounds: {\itshape``Sjo! Sjo!''} Little Lerrim, the Shoanti child the heroes saved from Gaedran Lamm's clutches, flies around the healer's neck. It looks like he's been doing well since he joined Thousand Bones and his men.\\

Thousand Bones invites the visitors in his {\itshape yurt} to talk to them. Quint clarifies that they are here to find out more about the evil that houses under the great mastaba in Korvosa. He fears that queen Ileosa has unearthed it and is now firmly under its influence. {\itshape``My friends, I have indeed heard speak of such an evil, and although I will not be able to tell you anything about it, I am glad you came to me first. You see, unlike our brethren in the other tribes, the Skoan-Quah are a peaceful people. Our willingness to talk to {\itshape tshamek} is not followed by the other Quah, who even think it brings shame to us. Still, they accept us as we watch over the souls of their dead. My words of peace do not reach their ears, instead they talk of war. They have heard stories of a city in flames, a king dying and a terrible disease decimating the people of Korvosa. They believe the time has come for them to reclaims the lands of their forefathers and descend once again to the green pastures in the south. Yet I say, war will not be good for us. My brethren refuse to see that even a crippled Korvosa is a formidable opponent. I deem it wiser to make friends of one's enemies, but the Sun clan only sees weakness and demands action.''}\\

{\itshape``As to your question about the evil under the mastaba, know that my people were once part of the Sklar-Quah -- the Sun clan. But since we were willing to forgive our enemies, we were no longer welcome among them. Over many generations our fathers found peace in these lands, but that does not go for the Sklar-Quah, who still believe that the lowlands of Korvosa are their only true home. Apart from guarding over the dead, my people also watch over the Shoanti's history. As such we know many tales from days past, but the knowledge of what was buried in the great mastaba is known only to a select few, the guardians of the word from the Sklar-Quah: the Sun shamans. In my long life I have only heard two of them mention this evil in what you call Korvosa. They called it {\itshape the Teeth of Midnight} , but that is all I know. If you want to learn more, you will have to ask the Sun Shamans yourself. Although I can speak to the Sun clan on your behalf, I will never be entrusted with their deepest secret, only to spill the information to you afterwards. Being {\itshape tshamek} you will never learn this secret either, even less so since you hail from the one place they hate most in the world. So if you want them to tell you, you will have to earn their respect and trust by throwing off your  {\itshape tshamek} mantle and becoming true  {\itshape nalharest} , not only to me, but to them as well! Only then will you stand a chance of learning about the evil under the great mastaba. We will reconvene tomorrow to ask the spirits for guidance. Until then you are welcome in our midst.}''\\

After the meeting with Thousand Bones, the companions join the tribesmen around the fire to share in a hearty meal and some stories. Sjo inquires whether the fact that he can call Sun shaman Yundur Firestorm his grandfather will earn him any favors among the Sklar-Quah, but the Skull people tell him that his non-Shoanti heritage is too much of a shame to make it so. Then he wonders if there have ever been people who have been freed of great shame and indeed, there is a legend of one who achieved just such a feat. His name was Skurak, a mighty Shoanti warrior, who stood accused of having murdered his own brother. Since family is sacred to the Shoanti, he was banished from the tribe and declared a {\itshape tshamek} . Skurak insisted that his brother's death was a hunting accident and swore to prove the truth by reclaiming his position among his people. He left for the hunting grounds of the deadly Cindermaw, the Clan-eater, a humongous sand worm who can swallow whole tribes at once. Skurak walked up to the beast, armed only with a dagger. Without fear he dove inside the monster's maw and carved his way out. Having been cleansed by the fires of the Cindermaw's acid belly, he was once again accepted by the Sun shamans, who declared he had been redeemed. The word  {\itshape tshamek} was never spoken to Skurak again, who passed into legend as Skurak the Reborn. Suddenly a new commotion stirs the camp. Seven Shoanti riders arrive, their red, orange and yellow colors identifying them as Burn Riders from the Sun clan. One very large and imposing tribesman leads them. {\itshape``That is Krojun \ldots Krojun Eats-What-He-Kills}'', Lerrim whispers. His rippling muscles are heavily covered in tattoos and war paint. He carries a mighty warhammer, an earthbreaker, and a Klar blade.\\

Several of the younger warriors from the Skull clan gather around these newcomers, regarding them with great respect. Krojun greets them with great gusto and welcomes their attention. Then he notices the companions and his body language changes. {\itshape``Why do the Skoan-Quah harbor tshamek intruders?}'' he growls in Shoanti.\\

{\itshape``These men are guests of Thousand Bones and thus guests of the tribe}'', jothka One-Life answers him. The look in the giant's eye does not show any acceptance, so Thousand Bones himself steps forward.\\

{\itshape``Tell me, Krojun Eats-What-He-Kills, since when do the Sklar-Quah decide whois is not''} Krojun snaps back at him: {\itshape``Those words do not change my question, old shaman. These tshamek bring trouble to the Cinderlands, you know that as well as I do. The future will reveal soon enough which of us is right, but to me they are just that: intruders.}''\\

{\itshape``You may be right, but I fear that the one bringing us trouble today is standing right in front of me. Tell me, Krojun Eats-What-He-Kills, why are you here?''} Thousand Bones replies.\\

The great barbarian's face glides into a wide grin, as he addresses not only to Thousand Bones, but all tribesmen: {\itshape``Why indeed am I here? You all know why. For you are not only the guardians of the dead, but also of knowledge. 82 winters our forefathers fought against the tshamek from Che-lee-axe (he spits on the ground) before being driven to these unhospitable lands. More than 200 years we have bided our time here, but now our enemies are weak. At last the time has come to strike back! I am here to gather your warriors. Once more the blood of our enemies will flow in honor of our forefathers, in honor of our totems, in honor of our sacred duty! (Some of the Skoan-Quah around Krojun let out a cheer.) We are SHOANTI and each quah will have to earn its own honor in battle or be doomed to live out its life as cowardly dogs! We have barely stretched our muscles and already tshamek are here, begging for peace.}''\\

This time Quint speaks up: {\itshape``Although we do not consider you our enemy and we wouldn't mind being friends, we are not here for peace, and we're even lees here to beg! You are mistaken, Burn Rider.}''\\

{\itshape``But it looks that I was not mistaken at all, Krojun Stormcrow}'', Thousand Bones takes over. {\itshape``You are the one who is here to bring trouble with your talk of war. How can you believe that spilling more Shoanti blood can ever honor our forefathers? Or do you think you can simply walk into Korvosa unopposed? The tshamek might be weakened, but that does not mean they are defenseless. Much blood will flow if you carry out your plans. We are indeed the guardians of the dead, but we do not invite death into our yurts.}'' Then Thousand Bones turns to his own tribesmen: {\itshape``I will not stand in your way if you want to join this warmonger on his insane quest. Each of you has to make his own decision and follow his own totem. But my totem tells me to walk the way of peace and these tshamek (he points at the party) are my guests on this road. You, Krojun, who are also a guest in our midst, will do well to remember that!''}\\

At first Krojun's muscles in his enormous neck tighten, as Quint warns him: {\itshape``Heed the words of your elder, warrior; listen to the shaman's wisdom.}''\\

Once more Krojun's face changes to a charismatic smile, as he answers to all gathered around him: {\itshape``As always the insight of age trumps my youthful enthusiasm and fire. I will honor these tshamek as your guests, Thousand Bones, as tradition requires. Who knows, they might even possess some courage since they are willing to travel so deep into Shoanti territory. Still, you will not mind that I test this courage in an innocent game of Sredna, now will you?}'' At that the barbarian digs up a leather strap from his pouch and dangles it before the companions. Thousand Bones gives the heroes a look and then shrugs, leaving it up to them to accept or refuse.\\

Lerrim explains to Sjo that Sredna is a game similar to tug-of-war. Two opponents tie a leather strap behind their heads and try to pull each other down until one of them falls of gives up. The game is innocent enough, as no one really gets hurt in it, but refusing the challenge would be considered a sign of cowardice, more even than losing.\\

